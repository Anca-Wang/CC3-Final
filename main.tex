\documentclass[11pt]{article}

    \usepackage[breakable]{tcolorbox}
    \usepackage{parskip} % Stop auto-indenting (to mimic markdown behaviour)
    

    % Basic figure setup, for now with no caption control since it's done
    % automatically by Pandoc (which extracts ![](path) syntax from Markdown).
    \usepackage{graphicx}
    % Maintain compatibility with old templates. Remove in nbconvert 6.0
    \let\Oldincludegraphics\includegraphics
    % Ensure that by default, figures have no caption (until we provide a
    % proper Figure object with a Caption API and a way to capture that
    % in the conversion process - todo).
    \usepackage{caption}
    \DeclareCaptionFormat{nocaption}{}
    \captionsetup{format=nocaption,aboveskip=0pt,belowskip=0pt}

    \usepackage{float}
    \floatplacement{figure}{H} % forces figures to be placed at the correct location
    \usepackage{xcolor} % Allow colors to be defined
    \usepackage{enumerate} % Needed for markdown enumerations to work
    \usepackage{geometry} % Used to adjust the document margins
    \usepackage{amsmath} % Equations
    \usepackage{amssymb} % Equations
    \usepackage{textcomp} % defines textquotesingle
    % Hack from http://tex.stackexchange.com/a/47451/13684:
    \AtBeginDocument{%
        \def\PYZsq{\textquotesingle}% Upright quotes in Pygmentized code
    }
    \usepackage{upquote} % Upright quotes for verbatim code
    \usepackage{eurosym} % defines \euro

    \usepackage{iftex}
    \ifPDFTeX
        \usepackage[T1]{fontenc}
        \IfFileExists{alphabeta.sty}{
              \usepackage{alphabeta}
          }{
              \usepackage[mathletters]{ucs}
              \usepackage[utf8x]{inputenc}
          }
    \else
        \usepackage{fontspec}
        \usepackage{unicode-math}
    \fi

    \usepackage{fancyvrb} % verbatim replacement that allows latex
    \usepackage{grffile} % extends the file name processing of package graphics
                         % to support a larger range
    \makeatletter % fix for old versions of grffile with XeLaTeX
    \@ifpackagelater{grffile}{2019/11/01}
    {
      % Do nothing on new versions
    }
    {
      \def\Gread@@xetex#1{%
        \IfFileExists{"\Gin@base".bb}%
        {\Gread@eps{\Gin@base.bb}}%
        {\Gread@@xetex@aux#1}%
      }
    }
    \makeatother
    \usepackage[Export]{adjustbox} % Used to constrain images to a maximum size
    \adjustboxset{max size={0.9\linewidth}{0.9\paperheight}}

    % The hyperref package gives us a pdf with properly built
    % internal navigation ('pdf bookmarks' for the table of contents,
    % internal cross-reference links, web links for URLs, etc.)
    \usepackage{hyperref}
    % The default LaTeX title has an obnoxious amount of whitespace. By default,
    % titling removes some of it. It also provides customization options.
    \usepackage{titling}
    \usepackage{longtable} % longtable support required by pandoc >1.10
    \usepackage{booktabs}  % table support for pandoc > 1.12.2
    \usepackage{array}     % table support for pandoc >= 2.11.3
    \usepackage{calc}      % table minipage width calculation for pandoc >= 2.11.1
    \usepackage[inline]{enumitem} % IRkernel/repr support (it uses the enumerate* environment)
    \usepackage[normalem]{ulem} % ulem is needed to support strikethroughs (\sout)
                                % normalem makes italics be italics, not underlines
    \usepackage{mathrsfs}
    

    
    % Colors for the hyperref package
    \definecolor{urlcolor}{rgb}{0,.145,.698}
    \definecolor{linkcolor}{rgb}{.71,0.21,0.01}
    \definecolor{citecolor}{rgb}{.12,.54,.11}

    % ANSI colors
    \definecolor{ansi-black}{HTML}{3E424D}
    \definecolor{ansi-black-intense}{HTML}{282C36}
    \definecolor{ansi-red}{HTML}{E75C58}
    \definecolor{ansi-red-intense}{HTML}{B22B31}
    \definecolor{ansi-green}{HTML}{00A250}
    \definecolor{ansi-green-intense}{HTML}{007427}
    \definecolor{ansi-yellow}{HTML}{DDB62B}
    \definecolor{ansi-yellow-intense}{HTML}{B27D12}
    \definecolor{ansi-blue}{HTML}{208FFB}
    \definecolor{ansi-blue-intense}{HTML}{0065CA}
    \definecolor{ansi-magenta}{HTML}{D160C4}
    \definecolor{ansi-magenta-intense}{HTML}{A03196}
    \definecolor{ansi-cyan}{HTML}{60C6C8}
    \definecolor{ansi-cyan-intense}{HTML}{258F8F}
    \definecolor{ansi-white}{HTML}{C5C1B4}
    \definecolor{ansi-white-intense}{HTML}{A1A6B2}
    \definecolor{ansi-default-inverse-fg}{HTML}{FFFFFF}
    \definecolor{ansi-default-inverse-bg}{HTML}{000000}

    % common color for the border for error outputs.
    \definecolor{outerrorbackground}{HTML}{FFDFDF}

    % commands and environments needed by pandoc snippets
    % extracted from the output of `pandoc -s`
    \providecommand{\tightlist}{%
      \setlength{\itemsep}{0pt}\setlength{\parskip}{0pt}}
    \DefineVerbatimEnvironment{Highlighting}{Verbatim}{commandchars=\\\{\}}
    % Add ',fontsize=\small' for more characters per line
    \newenvironment{Shaded}{}{}
    \newcommand{\KeywordTok}[1]{\textcolor[rgb]{0.00,0.44,0.13}{\textbf{{#1}}}}
    \newcommand{\DataTypeTok}[1]{\textcolor[rgb]{0.56,0.13,0.00}{{#1}}}
    \newcommand{\DecValTok}[1]{\textcolor[rgb]{0.25,0.63,0.44}{{#1}}}
    \newcommand{\BaseNTok}[1]{\textcolor[rgb]{0.25,0.63,0.44}{{#1}}}
    \newcommand{\FloatTok}[1]{\textcolor[rgb]{0.25,0.63,0.44}{{#1}}}
    \newcommand{\CharTok}[1]{\textcolor[rgb]{0.25,0.44,0.63}{{#1}}}
    \newcommand{\StringTok}[1]{\textcolor[rgb]{0.25,0.44,0.63}{{#1}}}
    \newcommand{\CommentTok}[1]{\textcolor[rgb]{0.38,0.63,0.69}{\textit{{#1}}}}
    \newcommand{\OtherTok}[1]{\textcolor[rgb]{0.00,0.44,0.13}{{#1}}}
    \newcommand{\AlertTok}[1]{\textcolor[rgb]{1.00,0.00,0.00}{\textbf{{#1}}}}
    \newcommand{\FunctionTok}[1]{\textcolor[rgb]{0.02,0.16,0.49}{{#1}}}
    \newcommand{\RegionMarkerTok}[1]{{#1}}
    \newcommand{\ErrorTok}[1]{\textcolor[rgb]{1.00,0.00,0.00}{\textbf{{#1}}}}
    \newcommand{\NormalTok}[1]{{#1}}

    % Additional commands for more recent versions of Pandoc
    \newcommand{\ConstantTok}[1]{\textcolor[rgb]{0.53,0.00,0.00}{{#1}}}
    \newcommand{\SpecialCharTok}[1]{\textcolor[rgb]{0.25,0.44,0.63}{{#1}}}
    \newcommand{\VerbatimStringTok}[1]{\textcolor[rgb]{0.25,0.44,0.63}{{#1}}}
    \newcommand{\SpecialStringTok}[1]{\textcolor[rgb]{0.73,0.40,0.53}{{#1}}}
    \newcommand{\ImportTok}[1]{{#1}}
    \newcommand{\DocumentationTok}[1]{\textcolor[rgb]{0.73,0.13,0.13}{\textit{{#1}}}}
    \newcommand{\AnnotationTok}[1]{\textcolor[rgb]{0.38,0.63,0.69}{\textbf{\textit{{#1}}}}}
    \newcommand{\CommentVarTok}[1]{\textcolor[rgb]{0.38,0.63,0.69}{\textbf{\textit{{#1}}}}}
    \newcommand{\VariableTok}[1]{\textcolor[rgb]{0.10,0.09,0.49}{{#1}}}
    \newcommand{\ControlFlowTok}[1]{\textcolor[rgb]{0.00,0.44,0.13}{\textbf{{#1}}}}
    \newcommand{\OperatorTok}[1]{\textcolor[rgb]{0.40,0.40,0.40}{{#1}}}
    \newcommand{\BuiltInTok}[1]{{#1}}
    \newcommand{\ExtensionTok}[1]{{#1}}
    \newcommand{\PreprocessorTok}[1]{\textcolor[rgb]{0.74,0.48,0.00}{{#1}}}
    \newcommand{\AttributeTok}[1]{\textcolor[rgb]{0.49,0.56,0.16}{{#1}}}
    \newcommand{\InformationTok}[1]{\textcolor[rgb]{0.38,0.63,0.69}{\textbf{\textit{{#1}}}}}
    \newcommand{\WarningTok}[1]{\textcolor[rgb]{0.38,0.63,0.69}{\textbf{\textit{{#1}}}}}


    % Define a nice break command that doesn't care if a line doesn't already
    % exist.
    \def\br{\hspace*{\fill} \\* }
    % Math Jax compatibility definitions
    \def\gt{>}
    \def\lt{<}
    \let\Oldtex\TeX
    \let\Oldlatex\LaTeX
    \renewcommand{\TeX}{\textrm{\Oldtex}}
    \renewcommand{\LaTeX}{\textrm{\Oldlatex}}
    % Document parameters
    % Document title
    \title{main}
    
    
    
    
    
    
    
% Pygments definitions
\makeatletter
\def\PY@reset{\let\PY@it=\relax \let\PY@bf=\relax%
    \let\PY@ul=\relax \let\PY@tc=\relax%
    \let\PY@bc=\relax \let\PY@ff=\relax}
\def\PY@tok#1{\csname PY@tok@#1\endcsname}
\def\PY@toks#1+{\ifx\relax#1\empty\else%
    \PY@tok{#1}\expandafter\PY@toks\fi}
\def\PY@do#1{\PY@bc{\PY@tc{\PY@ul{%
    \PY@it{\PY@bf{\PY@ff{#1}}}}}}}
\def\PY#1#2{\PY@reset\PY@toks#1+\relax+\PY@do{#2}}

\@namedef{PY@tok@w}{\def\PY@tc##1{\textcolor[rgb]{0.73,0.73,0.73}{##1}}}
\@namedef{PY@tok@c}{\let\PY@it=\textit\def\PY@tc##1{\textcolor[rgb]{0.24,0.48,0.48}{##1}}}
\@namedef{PY@tok@cp}{\def\PY@tc##1{\textcolor[rgb]{0.61,0.40,0.00}{##1}}}
\@namedef{PY@tok@k}{\let\PY@bf=\textbf\def\PY@tc##1{\textcolor[rgb]{0.00,0.50,0.00}{##1}}}
\@namedef{PY@tok@kp}{\def\PY@tc##1{\textcolor[rgb]{0.00,0.50,0.00}{##1}}}
\@namedef{PY@tok@kt}{\def\PY@tc##1{\textcolor[rgb]{0.69,0.00,0.25}{##1}}}
\@namedef{PY@tok@o}{\def\PY@tc##1{\textcolor[rgb]{0.40,0.40,0.40}{##1}}}
\@namedef{PY@tok@ow}{\let\PY@bf=\textbf\def\PY@tc##1{\textcolor[rgb]{0.67,0.13,1.00}{##1}}}
\@namedef{PY@tok@nb}{\def\PY@tc##1{\textcolor[rgb]{0.00,0.50,0.00}{##1}}}
\@namedef{PY@tok@nf}{\def\PY@tc##1{\textcolor[rgb]{0.00,0.00,1.00}{##1}}}
\@namedef{PY@tok@nc}{\let\PY@bf=\textbf\def\PY@tc##1{\textcolor[rgb]{0.00,0.00,1.00}{##1}}}
\@namedef{PY@tok@nn}{\let\PY@bf=\textbf\def\PY@tc##1{\textcolor[rgb]{0.00,0.00,1.00}{##1}}}
\@namedef{PY@tok@ne}{\let\PY@bf=\textbf\def\PY@tc##1{\textcolor[rgb]{0.80,0.25,0.22}{##1}}}
\@namedef{PY@tok@nv}{\def\PY@tc##1{\textcolor[rgb]{0.10,0.09,0.49}{##1}}}
\@namedef{PY@tok@no}{\def\PY@tc##1{\textcolor[rgb]{0.53,0.00,0.00}{##1}}}
\@namedef{PY@tok@nl}{\def\PY@tc##1{\textcolor[rgb]{0.46,0.46,0.00}{##1}}}
\@namedef{PY@tok@ni}{\let\PY@bf=\textbf\def\PY@tc##1{\textcolor[rgb]{0.44,0.44,0.44}{##1}}}
\@namedef{PY@tok@na}{\def\PY@tc##1{\textcolor[rgb]{0.41,0.47,0.13}{##1}}}
\@namedef{PY@tok@nt}{\let\PY@bf=\textbf\def\PY@tc##1{\textcolor[rgb]{0.00,0.50,0.00}{##1}}}
\@namedef{PY@tok@nd}{\def\PY@tc##1{\textcolor[rgb]{0.67,0.13,1.00}{##1}}}
\@namedef{PY@tok@s}{\def\PY@tc##1{\textcolor[rgb]{0.73,0.13,0.13}{##1}}}
\@namedef{PY@tok@sd}{\let\PY@it=\textit\def\PY@tc##1{\textcolor[rgb]{0.73,0.13,0.13}{##1}}}
\@namedef{PY@tok@si}{\let\PY@bf=\textbf\def\PY@tc##1{\textcolor[rgb]{0.64,0.35,0.47}{##1}}}
\@namedef{PY@tok@se}{\let\PY@bf=\textbf\def\PY@tc##1{\textcolor[rgb]{0.67,0.36,0.12}{##1}}}
\@namedef{PY@tok@sr}{\def\PY@tc##1{\textcolor[rgb]{0.64,0.35,0.47}{##1}}}
\@namedef{PY@tok@ss}{\def\PY@tc##1{\textcolor[rgb]{0.10,0.09,0.49}{##1}}}
\@namedef{PY@tok@sx}{\def\PY@tc##1{\textcolor[rgb]{0.00,0.50,0.00}{##1}}}
\@namedef{PY@tok@m}{\def\PY@tc##1{\textcolor[rgb]{0.40,0.40,0.40}{##1}}}
\@namedef{PY@tok@gh}{\let\PY@bf=\textbf\def\PY@tc##1{\textcolor[rgb]{0.00,0.00,0.50}{##1}}}
\@namedef{PY@tok@gu}{\let\PY@bf=\textbf\def\PY@tc##1{\textcolor[rgb]{0.50,0.00,0.50}{##1}}}
\@namedef{PY@tok@gd}{\def\PY@tc##1{\textcolor[rgb]{0.63,0.00,0.00}{##1}}}
\@namedef{PY@tok@gi}{\def\PY@tc##1{\textcolor[rgb]{0.00,0.52,0.00}{##1}}}
\@namedef{PY@tok@gr}{\def\PY@tc##1{\textcolor[rgb]{0.89,0.00,0.00}{##1}}}
\@namedef{PY@tok@ge}{\let\PY@it=\textit}
\@namedef{PY@tok@gs}{\let\PY@bf=\textbf}
\@namedef{PY@tok@gp}{\let\PY@bf=\textbf\def\PY@tc##1{\textcolor[rgb]{0.00,0.00,0.50}{##1}}}
\@namedef{PY@tok@go}{\def\PY@tc##1{\textcolor[rgb]{0.44,0.44,0.44}{##1}}}
\@namedef{PY@tok@gt}{\def\PY@tc##1{\textcolor[rgb]{0.00,0.27,0.87}{##1}}}
\@namedef{PY@tok@err}{\def\PY@bc##1{{\setlength{\fboxsep}{\string -\fboxrule}\fcolorbox[rgb]{1.00,0.00,0.00}{1,1,1}{\strut ##1}}}}
\@namedef{PY@tok@kc}{\let\PY@bf=\textbf\def\PY@tc##1{\textcolor[rgb]{0.00,0.50,0.00}{##1}}}
\@namedef{PY@tok@kd}{\let\PY@bf=\textbf\def\PY@tc##1{\textcolor[rgb]{0.00,0.50,0.00}{##1}}}
\@namedef{PY@tok@kn}{\let\PY@bf=\textbf\def\PY@tc##1{\textcolor[rgb]{0.00,0.50,0.00}{##1}}}
\@namedef{PY@tok@kr}{\let\PY@bf=\textbf\def\PY@tc##1{\textcolor[rgb]{0.00,0.50,0.00}{##1}}}
\@namedef{PY@tok@bp}{\def\PY@tc##1{\textcolor[rgb]{0.00,0.50,0.00}{##1}}}
\@namedef{PY@tok@fm}{\def\PY@tc##1{\textcolor[rgb]{0.00,0.00,1.00}{##1}}}
\@namedef{PY@tok@vc}{\def\PY@tc##1{\textcolor[rgb]{0.10,0.09,0.49}{##1}}}
\@namedef{PY@tok@vg}{\def\PY@tc##1{\textcolor[rgb]{0.10,0.09,0.49}{##1}}}
\@namedef{PY@tok@vi}{\def\PY@tc##1{\textcolor[rgb]{0.10,0.09,0.49}{##1}}}
\@namedef{PY@tok@vm}{\def\PY@tc##1{\textcolor[rgb]{0.10,0.09,0.49}{##1}}}
\@namedef{PY@tok@sa}{\def\PY@tc##1{\textcolor[rgb]{0.73,0.13,0.13}{##1}}}
\@namedef{PY@tok@sb}{\def\PY@tc##1{\textcolor[rgb]{0.73,0.13,0.13}{##1}}}
\@namedef{PY@tok@sc}{\def\PY@tc##1{\textcolor[rgb]{0.73,0.13,0.13}{##1}}}
\@namedef{PY@tok@dl}{\def\PY@tc##1{\textcolor[rgb]{0.73,0.13,0.13}{##1}}}
\@namedef{PY@tok@s2}{\def\PY@tc##1{\textcolor[rgb]{0.73,0.13,0.13}{##1}}}
\@namedef{PY@tok@sh}{\def\PY@tc##1{\textcolor[rgb]{0.73,0.13,0.13}{##1}}}
\@namedef{PY@tok@s1}{\def\PY@tc##1{\textcolor[rgb]{0.73,0.13,0.13}{##1}}}
\@namedef{PY@tok@mb}{\def\PY@tc##1{\textcolor[rgb]{0.40,0.40,0.40}{##1}}}
\@namedef{PY@tok@mf}{\def\PY@tc##1{\textcolor[rgb]{0.40,0.40,0.40}{##1}}}
\@namedef{PY@tok@mh}{\def\PY@tc##1{\textcolor[rgb]{0.40,0.40,0.40}{##1}}}
\@namedef{PY@tok@mi}{\def\PY@tc##1{\textcolor[rgb]{0.40,0.40,0.40}{##1}}}
\@namedef{PY@tok@il}{\def\PY@tc##1{\textcolor[rgb]{0.40,0.40,0.40}{##1}}}
\@namedef{PY@tok@mo}{\def\PY@tc##1{\textcolor[rgb]{0.40,0.40,0.40}{##1}}}
\@namedef{PY@tok@ch}{\let\PY@it=\textit\def\PY@tc##1{\textcolor[rgb]{0.24,0.48,0.48}{##1}}}
\@namedef{PY@tok@cm}{\let\PY@it=\textit\def\PY@tc##1{\textcolor[rgb]{0.24,0.48,0.48}{##1}}}
\@namedef{PY@tok@cpf}{\let\PY@it=\textit\def\PY@tc##1{\textcolor[rgb]{0.24,0.48,0.48}{##1}}}
\@namedef{PY@tok@c1}{\let\PY@it=\textit\def\PY@tc##1{\textcolor[rgb]{0.24,0.48,0.48}{##1}}}
\@namedef{PY@tok@cs}{\let\PY@it=\textit\def\PY@tc##1{\textcolor[rgb]{0.24,0.48,0.48}{##1}}}

\def\PYZbs{\char`\\}
\def\PYZus{\char`\_}
\def\PYZob{\char`\{}
\def\PYZcb{\char`\}}
\def\PYZca{\char`\^}
\def\PYZam{\char`\&}
\def\PYZlt{\char`\<}
\def\PYZgt{\char`\>}
\def\PYZsh{\char`\#}
\def\PYZpc{\char`\%}
\def\PYZdl{\char`\$}
\def\PYZhy{\char`\-}
\def\PYZsq{\char`\'}
\def\PYZdq{\char`\"}
\def\PYZti{\char`\~}
% for compatibility with earlier versions
\def\PYZat{@}
\def\PYZlb{[}
\def\PYZrb{]}
\makeatother


    % For linebreaks inside Verbatim environment from package fancyvrb.
    \makeatletter
        \newbox\Wrappedcontinuationbox
        \newbox\Wrappedvisiblespacebox
        \newcommand*\Wrappedvisiblespace {\textcolor{red}{\textvisiblespace}}
        \newcommand*\Wrappedcontinuationsymbol {\textcolor{red}{\llap{\tiny$\m@th\hookrightarrow$}}}
        \newcommand*\Wrappedcontinuationindent {3ex }
        \newcommand*\Wrappedafterbreak {\kern\Wrappedcontinuationindent\copy\Wrappedcontinuationbox}
        % Take advantage of the already applied Pygments mark-up to insert
        % potential linebreaks for TeX processing.
        %        {, <, #, %, $, ' and ": go to next line.
        %        _, }, ^, &, >, - and ~: stay at end of broken line.
        % Use of \textquotesingle for straight quote.
        \newcommand*\Wrappedbreaksatspecials {%
            \def\PYGZus{\discretionary{\char`\_}{\Wrappedafterbreak}{\char`\_}}%
            \def\PYGZob{\discretionary{}{\Wrappedafterbreak\char`\{}{\char`\{}}%
            \def\PYGZcb{\discretionary{\char`\}}{\Wrappedafterbreak}{\char`\}}}%
            \def\PYGZca{\discretionary{\char`\^}{\Wrappedafterbreak}{\char`\^}}%
            \def\PYGZam{\discretionary{\char`\&}{\Wrappedafterbreak}{\char`\&}}%
            \def\PYGZlt{\discretionary{}{\Wrappedafterbreak\char`\<}{\char`\<}}%
            \def\PYGZgt{\discretionary{\char`\>}{\Wrappedafterbreak}{\char`\>}}%
            \def\PYGZsh{\discretionary{}{\Wrappedafterbreak\char`\#}{\char`\#}}%
            \def\PYGZpc{\discretionary{}{\Wrappedafterbreak\char`\%}{\char`\%}}%
            \def\PYGZdl{\discretionary{}{\Wrappedafterbreak\char`\$}{\char`\$}}%
            \def\PYGZhy{\discretionary{\char`\-}{\Wrappedafterbreak}{\char`\-}}%
            \def\PYGZsq{\discretionary{}{\Wrappedafterbreak\textquotesingle}{\textquotesingle}}%
            \def\PYGZdq{\discretionary{}{\Wrappedafterbreak\char`\"}{\char`\"}}%
            \def\PYGZti{\discretionary{\char`\~}{\Wrappedafterbreak}{\char`\~}}%
        }
        % Some characters . , ; ? ! / are not pygmentized.
        % This macro makes them "active" and they will insert potential linebreaks
        \newcommand*\Wrappedbreaksatpunct {%
            \lccode`\~`\.\lowercase{\def~}{\discretionary{\hbox{\char`\.}}{\Wrappedafterbreak}{\hbox{\char`\.}}}%
            \lccode`\~`\,\lowercase{\def~}{\discretionary{\hbox{\char`\,}}{\Wrappedafterbreak}{\hbox{\char`\,}}}%
            \lccode`\~`\;\lowercase{\def~}{\discretionary{\hbox{\char`\;}}{\Wrappedafterbreak}{\hbox{\char`\;}}}%
            \lccode`\~`\:\lowercase{\def~}{\discretionary{\hbox{\char`\:}}{\Wrappedafterbreak}{\hbox{\char`\:}}}%
            \lccode`\~`\?\lowercase{\def~}{\discretionary{\hbox{\char`\?}}{\Wrappedafterbreak}{\hbox{\char`\?}}}%
            \lccode`\~`\!\lowercase{\def~}{\discretionary{\hbox{\char`\!}}{\Wrappedafterbreak}{\hbox{\char`\!}}}%
            \lccode`\~`\/\lowercase{\def~}{\discretionary{\hbox{\char`\/}}{\Wrappedafterbreak}{\hbox{\char`\/}}}%
            \catcode`\.\active
            \catcode`\,\active
            \catcode`\;\active
            \catcode`\:\active
            \catcode`\?\active
            \catcode`\!\active
            \catcode`\/\active
            \lccode`\~`\~
        }
    \makeatother

    \let\OriginalVerbatim=\Verbatim
    \makeatletter
    \renewcommand{\Verbatim}[1][1]{%
        %\parskip\z@skip
        \sbox\Wrappedcontinuationbox {\Wrappedcontinuationsymbol}%
        \sbox\Wrappedvisiblespacebox {\FV@SetupFont\Wrappedvisiblespace}%
        \def\FancyVerbFormatLine ##1{\hsize\linewidth
            \vtop{\raggedright\hyphenpenalty\z@\exhyphenpenalty\z@
                \doublehyphendemerits\z@\finalhyphendemerits\z@
                \strut ##1\strut}%
        }%
        % If the linebreak is at a space, the latter will be displayed as visible
        % space at end of first line, and a continuation symbol starts next line.
        % Stretch/shrink are however usually zero for typewriter font.
        \def\FV@Space {%
            \nobreak\hskip\z@ plus\fontdimen3\font minus\fontdimen4\font
            \discretionary{\copy\Wrappedvisiblespacebox}{\Wrappedafterbreak}
            {\kern\fontdimen2\font}%
        }%

        % Allow breaks at special characters using \PYG... macros.
        \Wrappedbreaksatspecials
        % Breaks at punctuation characters . , ; ? ! and / need catcode=\active
        \OriginalVerbatim[#1,codes*=\Wrappedbreaksatpunct]%
    }
    \makeatother

    % Exact colors from NB
    \definecolor{incolor}{HTML}{303F9F}
    \definecolor{outcolor}{HTML}{D84315}
    \definecolor{cellborder}{HTML}{CFCFCF}
    \definecolor{cellbackground}{HTML}{F7F7F7}

    % prompt
    \makeatletter
    \newcommand{\boxspacing}{\kern\kvtcb@left@rule\kern\kvtcb@boxsep}
    \makeatother
    \newcommand{\prompt}[4]{
        {\ttfamily\llap{{\color{#2}[#3]:\hspace{3pt}#4}}\vspace{-\baselineskip}}
    }
    

    
    % Prevent overflowing lines due to hard-to-break entities
    \sloppy
    % Setup hyperref package
    \hypersetup{
      breaklinks=true,  % so long urls are correctly broken across lines
      colorlinks=true,
      urlcolor=urlcolor,
      linkcolor=linkcolor,
      citecolor=citecolor,
      }
    % Slightly bigger margins than the latex defaults
    
    \geometry{verbose,tmargin=1in,bmargin=1in,lmargin=1in,rmargin=1in}
    
    

\begin{document}
    
    \maketitle
    
    

    
    \begin{tcolorbox}[breakable, size=fbox, boxrule=1pt, pad at break*=1mm,colback=cellbackground, colframe=cellborder]
\prompt{In}{incolor}{204}{\boxspacing}
\begin{Verbatim}[commandchars=\\\{\}]
\PY{c+c1}{\PYZsh{}credit: https://www.kaggle.com/code/shivanshuman/vgg19\PYZhy{}transfer\PYZhy{}learning}

\PY{k+kn}{import} \PY{n+nn}{pandas} \PY{k}{as} \PY{n+nn}{pd} \PY{c+c1}{\PYZsh{} data processing, CSV file I/O (e.g. pd.read\PYZus{}csv)}
\PY{k+kn}{import} \PY{n+nn}{numpy} \PY{k}{as} \PY{n+nn}{np}
\PY{k+kn}{import} \PY{n+nn}{os}
\PY{k+kn}{import} \PY{n+nn}{tensorflow} \PY{k}{as} \PY{n+nn}{tf}
\PY{n}{tf}\PY{o}{.}\PY{n}{compat}\PY{o}{.}\PY{n}{v1}\PY{o}{.}\PY{n}{disable\PYZus{}eager\PYZus{}execution}\PY{p}{(}\PY{p}{)}
\PY{k+kn}{from} \PY{n+nn}{keras} \PY{k+kn}{import} \PY{n}{backend} \PY{k}{as} \PY{n}{K}
\PY{k+kn}{from} \PY{n+nn}{keras}\PY{n+nn}{.}\PY{n+nn}{utils} \PY{k+kn}{import} \PY{n}{image\PYZus{}utils}
\PY{k+kn}{import} \PY{n+nn}{matplotlib}\PY{n+nn}{.}\PY{n+nn}{pyplot} \PY{k}{as} \PY{n+nn}{plt}
\PY{k+kn}{from} \PY{n+nn}{keras}\PY{n+nn}{.}\PY{n+nn}{applications} \PY{k+kn}{import} \PY{n}{vgg19}
\PY{k+kn}{from} \PY{n+nn}{keras}\PY{n+nn}{.}\PY{n+nn}{models} \PY{k+kn}{import} \PY{n}{Model}
\PY{c+c1}{\PYZsh{}from keras import optimizers}
\PY{k+kn}{from} \PY{n+nn}{scipy}\PY{n+nn}{.}\PY{n+nn}{optimize} \PY{k+kn}{import} \PY{n}{fmin\PYZus{}l\PYZus{}bfgs\PYZus{}b}
\PY{c+c1}{\PYZsh{}from keras.applications.vgg19 import VGG19}
\PY{c+c1}{\PYZsh{}vgg19\PYZus{}weights = \PYZsq{}../input/vgg19/vgg19\PYZus{}weights\PYZus{}tf\PYZus{}dim\PYZus{}ordering\PYZus{}tf\PYZus{}kernels\PYZus{}notop.h5\PYZsq{}}
\PY{c+c1}{\PYZsh{}vgg19 = VGG19(include\PYZus{}top = False, weights=vgg19\PYZus{}weights)}
\PY{n+nb}{print}\PY{p}{(}\PY{n}{os}\PY{o}{.}\PY{n}{listdir}\PY{p}{(}\PY{l+s+s2}{\PYZdq{}}\PY{l+s+s2}{./input/}\PY{l+s+s2}{\PYZdq{}}\PY{p}{)}\PY{p}{)}
\PY{c+c1}{\PYZsh{} Any results you write to the current directory are saved as output.}
\end{Verbatim}
\end{tcolorbox}

    \begin{Verbatim}[commandchars=\\\{\}]
['.DS\_Store', '777.JPG', 'three.JPG', '666.JPG', 'two.JPG', '2.webp', '11.jpg',
'2222.png', '1111.png', '640.jpg', '333.jpg', 'b.jpeg', '111111.jpeg', '7.jpg',
'444.jpeg', 'style.png', '3.jpeg']
    \end{Verbatim}

    \begin{tcolorbox}[breakable, size=fbox, boxrule=1pt, pad at break*=1mm,colback=cellbackground, colframe=cellborder]
\prompt{In}{incolor}{205}{\boxspacing}
\begin{Verbatim}[commandchars=\\\{\}]
\PY{n}{StylePath} \PY{o}{=} \PY{l+s+s1}{\PYZsq{}}\PY{l+s+s1}{./input/}\PY{l+s+s1}{\PYZsq{}}
\PY{n}{ContentPath} \PY{o}{=} \PY{l+s+s1}{\PYZsq{}}\PY{l+s+s1}{./input/}\PY{l+s+s1}{\PYZsq{}}
\end{Verbatim}
\end{tcolorbox}

    \begin{tcolorbox}[breakable, size=fbox, boxrule=1pt, pad at break*=1mm,colback=cellbackground, colframe=cellborder]
\prompt{In}{incolor}{206}{\boxspacing}
\begin{Verbatim}[commandchars=\\\{\}]
\PY{n}{style\PYZus{}image\PYZus{}path} \PY{o}{=} \PY{n}{StylePath}\PY{o}{+}\PY{l+s+s1}{\PYZsq{}}\PY{l+s+s1}{style.png}\PY{l+s+s1}{\PYZsq{}}
\PY{n}{base\PYZus{}image\PYZus{}path} \PY{o}{=} \PY{n}{ContentPath}\PY{o}{+}\PY{l+s+s1}{\PYZsq{}}\PY{l+s+s1}{444.jpeg}\PY{l+s+s1}{\PYZsq{}}
\end{Verbatim}
\end{tcolorbox}

    \begin{tcolorbox}[breakable, size=fbox, boxrule=1pt, pad at break*=1mm,colback=cellbackground, colframe=cellborder]
\prompt{In}{incolor}{207}{\boxspacing}
\begin{Verbatim}[commandchars=\\\{\}]
\PY{c+c1}{\PYZsh{} dimensions of the generated picture.}
\PY{n}{width}\PY{p}{,} \PY{n}{height} \PY{o}{=} \PY{n}{image\PYZus{}utils}\PY{o}{.}\PY{n}{load\PYZus{}img}\PY{p}{(}\PY{n}{base\PYZus{}image\PYZus{}path}\PY{p}{)}\PY{o}{.}\PY{n}{size}
\PY{n}{img\PYZus{}nrows} \PY{o}{=} \PY{l+m+mi}{400}
\PY{n}{img\PYZus{}ncols} \PY{o}{=} \PY{n+nb}{int}\PY{p}{(}\PY{n}{width} \PY{o}{*} \PY{n}{img\PYZus{}nrows} \PY{o}{/} \PY{n}{height}\PY{p}{)}
\end{Verbatim}
\end{tcolorbox}

    \begin{tcolorbox}[breakable, size=fbox, boxrule=1pt, pad at break*=1mm,colback=cellbackground, colframe=cellborder]
\prompt{In}{incolor}{208}{\boxspacing}
\begin{Verbatim}[commandchars=\\\{\}]
\PY{k}{def} \PY{n+nf}{preprocess\PYZus{}image}\PY{p}{(}\PY{n}{image\PYZus{}path}\PY{p}{)}\PY{p}{:}
    \PY{k+kn}{from} \PY{n+nn}{keras}\PY{n+nn}{.}\PY{n+nn}{applications} \PY{k+kn}{import} \PY{n}{vgg19}
    \PY{n}{img} \PY{o}{=} \PY{n}{image\PYZus{}utils}\PY{o}{.}\PY{n}{load\PYZus{}img}\PY{p}{(}\PY{n}{image\PYZus{}path}\PY{p}{,} \PY{n}{target\PYZus{}size}\PY{o}{=}\PY{p}{(}\PY{n}{img\PYZus{}nrows}\PY{p}{,} \PY{n}{img\PYZus{}ncols}\PY{p}{)}\PY{p}{)}
    \PY{n}{img} \PY{o}{=} \PY{n}{image\PYZus{}utils}\PY{o}{.}\PY{n}{img\PYZus{}to\PYZus{}array}\PY{p}{(}\PY{n}{img}\PY{p}{)}
    \PY{n}{img} \PY{o}{=} \PY{n}{np}\PY{o}{.}\PY{n}{expand\PYZus{}dims}\PY{p}{(}\PY{n}{img}\PY{p}{,} \PY{n}{axis}\PY{o}{=}\PY{l+m+mi}{0}\PY{p}{)}
    \PY{n}{img} \PY{o}{=} \PY{n}{vgg19}\PY{o}{.}\PY{n}{preprocess\PYZus{}input}\PY{p}{(}\PY{n}{img}\PY{p}{)}
    \PY{k}{return} \PY{n}{img}
\end{Verbatim}
\end{tcolorbox}

    \begin{tcolorbox}[breakable, size=fbox, boxrule=1pt, pad at break*=1mm,colback=cellbackground, colframe=cellborder]
\prompt{In}{incolor}{209}{\boxspacing}
\begin{Verbatim}[commandchars=\\\{\}]
\PY{n}{plt}\PY{o}{.}\PY{n}{figure}\PY{p}{(}\PY{p}{)}
\PY{n}{plt}\PY{o}{.}\PY{n}{title}\PY{p}{(}\PY{l+s+s2}{\PYZdq{}}\PY{l+s+s2}{Base Image}\PY{l+s+s2}{\PYZdq{}}\PY{p}{,}\PY{n}{fontsize}\PY{o}{=}\PY{l+m+mi}{20}\PY{p}{)}
\PY{n}{img1} \PY{o}{=} \PY{n}{image\PYZus{}utils}\PY{o}{.}\PY{n}{load\PYZus{}img}\PY{p}{(}\PY{n}{base\PYZus{}image\PYZus{}path}\PY{p}{)}
\PY{n}{plt}\PY{o}{.}\PY{n}{imshow}\PY{p}{(}\PY{n}{img1}\PY{p}{)}
\end{Verbatim}
\end{tcolorbox}

            \begin{tcolorbox}[breakable, size=fbox, boxrule=.5pt, pad at break*=1mm, opacityfill=0]
\prompt{Out}{outcolor}{209}{\boxspacing}
\begin{Verbatim}[commandchars=\\\{\}]
<matplotlib.image.AxesImage at 0x29b29c490>
\end{Verbatim}
\end{tcolorbox}
        
    \begin{center}
    \adjustimage{max size={0.9\linewidth}{0.9\paperheight}}{output_5_1.png}
    \end{center}
    { \hspace*{\fill} \\}
    
    \begin{tcolorbox}[breakable, size=fbox, boxrule=1pt, pad at break*=1mm,colback=cellbackground, colframe=cellborder]
\prompt{In}{incolor}{210}{\boxspacing}
\begin{Verbatim}[commandchars=\\\{\}]
\PY{n}{plt}\PY{o}{.}\PY{n}{figure}\PY{p}{(}\PY{p}{)}
\PY{n}{plt}\PY{o}{.}\PY{n}{title}\PY{p}{(}\PY{l+s+s2}{\PYZdq{}}\PY{l+s+s2}{Style Image}\PY{l+s+s2}{\PYZdq{}}\PY{p}{,}\PY{n}{fontsize}\PY{o}{=}\PY{l+m+mi}{20}\PY{p}{)}
\PY{n}{img1} \PY{o}{=} \PY{n}{image\PYZus{}utils}\PY{o}{.}\PY{n}{load\PYZus{}img}\PY{p}{(}\PY{n}{style\PYZus{}image\PYZus{}path}\PY{p}{)}
\PY{n}{plt}\PY{o}{.}\PY{n}{imshow}\PY{p}{(}\PY{n}{img1}\PY{p}{)}
\end{Verbatim}
\end{tcolorbox}

            \begin{tcolorbox}[breakable, size=fbox, boxrule=.5pt, pad at break*=1mm, opacityfill=0]
\prompt{Out}{outcolor}{210}{\boxspacing}
\begin{Verbatim}[commandchars=\\\{\}]
<matplotlib.image.AxesImage at 0x2cc46d400>
\end{Verbatim}
\end{tcolorbox}
        
    \begin{center}
    \adjustimage{max size={0.9\linewidth}{0.9\paperheight}}{output_6_1.png}
    \end{center}
    { \hspace*{\fill} \\}
    
    \begin{tcolorbox}[breakable, size=fbox, boxrule=1pt, pad at break*=1mm,colback=cellbackground, colframe=cellborder]
\prompt{In}{incolor}{211}{\boxspacing}
\begin{Verbatim}[commandchars=\\\{\}]
\PY{c+c1}{\PYZsh{} get tensor representations of our images}
\PY{n}{base\PYZus{}image} \PY{o}{=} \PY{n}{K}\PY{o}{.}\PY{n}{variable}\PY{p}{(}\PY{n}{preprocess\PYZus{}image}\PY{p}{(}\PY{n}{base\PYZus{}image\PYZus{}path}\PY{p}{)}\PY{p}{)}
\PY{n}{style\PYZus{}reference\PYZus{}image} \PY{o}{=} \PY{n}{K}\PY{o}{.}\PY{n}{variable}\PY{p}{(}\PY{n}{preprocess\PYZus{}image}\PY{p}{(}\PY{n}{style\PYZus{}image\PYZus{}path}\PY{p}{)}\PY{p}{)}
\end{Verbatim}
\end{tcolorbox}

    \begin{tcolorbox}[breakable, size=fbox, boxrule=1pt, pad at break*=1mm,colback=cellbackground, colframe=cellborder]
\prompt{In}{incolor}{212}{\boxspacing}
\begin{Verbatim}[commandchars=\\\{\}]
\PY{n}{K}\PY{o}{.}\PY{n}{image\PYZus{}data\PYZus{}format}\PY{p}{(}\PY{p}{)}
\end{Verbatim}
\end{tcolorbox}

            \begin{tcolorbox}[breakable, size=fbox, boxrule=.5pt, pad at break*=1mm, opacityfill=0]
\prompt{Out}{outcolor}{212}{\boxspacing}
\begin{Verbatim}[commandchars=\\\{\}]
'channels\_last'
\end{Verbatim}
\end{tcolorbox}
        
    \begin{tcolorbox}[breakable, size=fbox, boxrule=1pt, pad at break*=1mm,colback=cellbackground, colframe=cellborder]
\prompt{In}{incolor}{213}{\boxspacing}
\begin{Verbatim}[commandchars=\\\{\}]
\PY{c+c1}{\PYZsh{} this will contain our generated image}
\PY{k}{if} \PY{n}{K}\PY{o}{.}\PY{n}{image\PYZus{}data\PYZus{}format}\PY{p}{(}\PY{p}{)} \PY{o}{==} \PY{l+s+s1}{\PYZsq{}}\PY{l+s+s1}{channels\PYZus{}first}\PY{l+s+s1}{\PYZsq{}}\PY{p}{:}
    \PY{n}{combination\PYZus{}image} \PY{o}{=} \PY{n}{K}\PY{o}{.}\PY{n}{placeholder}\PY{p}{(}\PY{p}{(}\PY{l+m+mi}{1}\PY{p}{,}\PY{l+m+mi}{3}\PY{p}{,}\PY{n}{img\PYZus{}nrows}\PY{p}{,} \PY{n}{img\PYZus{}ncols}\PY{p}{)}\PY{p}{)}
\PY{k}{else}\PY{p}{:}
    \PY{n}{combination\PYZus{}image} \PY{o}{=} \PY{n}{K}\PY{o}{.}\PY{n}{placeholder}\PY{p}{(}\PY{p}{(}\PY{l+m+mi}{1}\PY{p}{,}\PY{n}{img\PYZus{}nrows}\PY{p}{,} \PY{n}{img\PYZus{}ncols}\PY{p}{,}\PY{l+m+mi}{3}\PY{p}{)}\PY{p}{)}
\end{Verbatim}
\end{tcolorbox}

    \begin{tcolorbox}[breakable, size=fbox, boxrule=1pt, pad at break*=1mm,colback=cellbackground, colframe=cellborder]
\prompt{In}{incolor}{214}{\boxspacing}
\begin{Verbatim}[commandchars=\\\{\}]
\PY{c+c1}{\PYZsh{} combine the 3 images into a single Keras tensor}
\PY{n}{input\PYZus{}tensor} \PY{o}{=} \PY{n}{K}\PY{o}{.}\PY{n}{concatenate}\PY{p}{(}\PY{p}{[}\PY{n}{base\PYZus{}image}\PY{p}{,}
                              \PY{n}{style\PYZus{}reference\PYZus{}image}\PY{p}{,}
                              \PY{n}{combination\PYZus{}image}
                              \PY{p}{]}\PY{p}{,} \PY{n}{axis}\PY{o}{=}\PY{l+m+mi}{0}\PY{p}{)}
\end{Verbatim}
\end{tcolorbox}

    \begin{tcolorbox}[breakable, size=fbox, boxrule=1pt, pad at break*=1mm,colback=cellbackground, colframe=cellborder]
\prompt{In}{incolor}{215}{\boxspacing}
\begin{Verbatim}[commandchars=\\\{\}]
\PY{c+c1}{\PYZsh{} build the VGG19 network with our 3 images as input}
\PY{c+c1}{\PYZsh{} the model will be loaded with pre\PYZhy{}trained ImageNet weights}
\PY{k+kn}{from} \PY{n+nn}{keras}\PY{n+nn}{.}\PY{n+nn}{applications}\PY{n+nn}{.}\PY{n+nn}{vgg19} \PY{k+kn}{import} \PY{n}{VGG19}
\PY{n}{vgg19\PYZus{}weights} \PY{o}{=} \PY{l+s+s1}{\PYZsq{}}\PY{l+s+s1}{./vgg19\PYZus{}weights\PYZus{}tf\PYZus{}dim\PYZus{}ordering\PYZus{}tf\PYZus{}kernels\PYZus{}notop.h5}\PY{l+s+s1}{\PYZsq{}}
\PY{n}{model} \PY{o}{=} \PY{n}{VGG19}\PY{p}{(}\PY{n}{input\PYZus{}tensor}\PY{o}{=}\PY{n}{input\PYZus{}tensor}\PY{p}{,}
              \PY{n}{include\PYZus{}top} \PY{o}{=} \PY{k+kc}{False}\PY{p}{,}
              \PY{n}{weights}\PY{o}{=}\PY{n}{vgg19\PYZus{}weights}\PY{p}{)}
\PY{c+c1}{\PYZsh{}model = vgg19.VGG19(input\PYZus{}tensor=input\PYZus{}tensor,}
\PY{c+c1}{\PYZsh{}                    weights=\PYZsq{}imagenet\PYZsq{}, include\PYZus{}top=False)}
\PY{n+nb}{print}\PY{p}{(}\PY{l+s+s1}{\PYZsq{}}\PY{l+s+s1}{Model loaded.}\PY{l+s+s1}{\PYZsq{}}\PY{p}{)}
\end{Verbatim}
\end{tcolorbox}

    \begin{Verbatim}[commandchars=\\\{\}]
2023-06-16 14:56:03.856871: W tensorflow/c/c\_api.cc:300] Operation
'\{name:'block5\_conv1\_8/kernel/Assign' id:8225 op device:\{requested: '',
assigned: ''\} def:\{\{\{node block5\_conv1\_8/kernel/Assign\}\} =
AssignVariableOp[\_has\_manual\_control\_dependencies=true, dtype=DT\_FLOAT,
validate\_shape=false](block5\_conv1\_8/kernel,
block5\_conv1\_8/kernel/Initializer/stateless\_random\_uniform)\}\}' was changed by
setting attribute after it was run by a session. This mutation will have no
effect, and will trigger an error in the future. Either don't modify nodes after
running them or create a new session.
    \end{Verbatim}

    \begin{Verbatim}[commandchars=\\\{\}]
Model loaded.
    \end{Verbatim}

    \begin{tcolorbox}[breakable, size=fbox, boxrule=1pt, pad at break*=1mm,colback=cellbackground, colframe=cellborder]
\prompt{In}{incolor}{216}{\boxspacing}
\begin{Verbatim}[commandchars=\\\{\}]
\PY{c+c1}{\PYZsh{} Content layer where will pull our feature maps}
\PY{n}{content\PYZus{}layers} \PY{o}{=} \PY{p}{[}\PY{l+s+s1}{\PYZsq{}}\PY{l+s+s1}{block5\PYZus{}conv2}\PY{l+s+s1}{\PYZsq{}}\PY{p}{]} 

\PY{c+c1}{\PYZsh{} Style layer we are interested in}
\PY{n}{style\PYZus{}layers} \PY{o}{=} \PY{p}{[}\PY{l+s+s1}{\PYZsq{}}\PY{l+s+s1}{block1\PYZus{}conv1}\PY{l+s+s1}{\PYZsq{}}\PY{p}{,}
                \PY{l+s+s1}{\PYZsq{}}\PY{l+s+s1}{block2\PYZus{}conv1}\PY{l+s+s1}{\PYZsq{}}\PY{p}{,}
                \PY{l+s+s1}{\PYZsq{}}\PY{l+s+s1}{block3\PYZus{}conv1}\PY{l+s+s1}{\PYZsq{}}\PY{p}{,} 
                \PY{l+s+s1}{\PYZsq{}}\PY{l+s+s1}{block4\PYZus{}conv1}\PY{l+s+s1}{\PYZsq{}}\PY{p}{,}
                \PY{l+s+s1}{\PYZsq{}}\PY{l+s+s1}{block5\PYZus{}conv1}\PY{l+s+s1}{\PYZsq{}}
               \PY{p}{]}

\PY{n}{num\PYZus{}content\PYZus{}layers} \PY{o}{=} \PY{n+nb}{len}\PY{p}{(}\PY{n}{content\PYZus{}layers}\PY{p}{)}
\PY{n}{num\PYZus{}style\PYZus{}layers} \PY{o}{=} \PY{n+nb}{len}\PY{p}{(}\PY{n}{style\PYZus{}layers}\PY{p}{)}
\end{Verbatim}
\end{tcolorbox}

    \begin{tcolorbox}[breakable, size=fbox, boxrule=1pt, pad at break*=1mm,colback=cellbackground, colframe=cellborder]
\prompt{In}{incolor}{217}{\boxspacing}
\begin{Verbatim}[commandchars=\\\{\}]
\PY{n}{outputs\PYZus{}dict} \PY{o}{=} \PY{n+nb}{dict}\PY{p}{(}\PY{p}{[}\PY{p}{(}\PY{n}{layer}\PY{o}{.}\PY{n}{name}\PY{p}{,} \PY{n}{layer}\PY{o}{.}\PY{n}{output}\PY{p}{)} \PY{k}{for} \PY{n}{layer} \PY{o+ow}{in} \PY{n}{model}\PY{o}{.}\PY{n}{layers}\PY{p}{]}\PY{p}{)}
\PY{n+nb}{print}\PY{p}{(}\PY{n}{outputs\PYZus{}dict}\PY{p}{[}\PY{l+s+s1}{\PYZsq{}}\PY{l+s+s1}{block5\PYZus{}conv2}\PY{l+s+s1}{\PYZsq{}}\PY{p}{]}\PY{p}{)}
\end{Verbatim}
\end{tcolorbox}

    \begin{Verbatim}[commandchars=\\\{\}]
Tensor("block5\_conv2\_8/Relu:0", shape=(3, 25, 43, 512), dtype=float32)
    \end{Verbatim}

    \begin{tcolorbox}[breakable, size=fbox, boxrule=1pt, pad at break*=1mm,colback=cellbackground, colframe=cellborder]
\prompt{In}{incolor}{218}{\boxspacing}
\begin{Verbatim}[commandchars=\\\{\}]
\PY{c+c1}{\PYZsh{} an auxiliary loss function}
\PY{c+c1}{\PYZsh{} designed to maintain the \PYZdq{}content\PYZdq{} of the}
\PY{c+c1}{\PYZsh{} base image in the generated image}
\PY{k}{def} \PY{n+nf}{get\PYZus{}content\PYZus{}loss}\PY{p}{(}\PY{n}{base\PYZus{}content}\PY{p}{,} \PY{n}{target}\PY{p}{)}\PY{p}{:}
    \PY{k}{return} \PY{n}{K}\PY{o}{.}\PY{n}{sum}\PY{p}{(}\PY{n}{K}\PY{o}{.}\PY{n}{square}\PY{p}{(}\PY{n}{target} \PY{o}{\PYZhy{}} \PY{n}{base\PYZus{}content}\PY{p}{)}\PY{p}{)}
\end{Verbatim}
\end{tcolorbox}

    \begin{tcolorbox}[breakable, size=fbox, boxrule=1pt, pad at break*=1mm,colback=cellbackground, colframe=cellborder]
\prompt{In}{incolor}{219}{\boxspacing}
\begin{Verbatim}[commandchars=\\\{\}]
\PY{k+kn}{import} \PY{n+nn}{tensorflow} \PY{k}{as} \PY{n+nn}{tf}
\PY{c+c1}{\PYZsh{} the gram matrix of an image tensor (feature\PYZhy{}wise outer product)}
\PY{k}{def} \PY{n+nf}{gram\PYZus{}matrix}\PY{p}{(}\PY{n}{input\PYZus{}tensor}\PY{p}{)}\PY{p}{:}
    \PY{k}{assert} \PY{n}{K}\PY{o}{.}\PY{n}{ndim}\PY{p}{(}\PY{n}{input\PYZus{}tensor}\PY{p}{)}\PY{o}{==}\PY{l+m+mi}{3}
    \PY{c+c1}{\PYZsh{}if K.image\PYZus{}data\PYZus{}format() == \PYZsq{}channels\PYZus{}first\PYZsq{}:}
    \PY{c+c1}{\PYZsh{}    features = K.batch\PYZus{}flatten(input\PYZus{}tensor)}
    \PY{c+c1}{\PYZsh{}else:}
    \PY{c+c1}{\PYZsh{}    features = K.batch\PYZus{}flatten(K.permute\PYZus{}dimensions(input\PYZus{}tensor,(2,0,1)))}
    \PY{c+c1}{\PYZsh{}gram = K.dot(features, K.transpose(features))}
    \PY{n}{channels} \PY{o}{=} \PY{n+nb}{int}\PY{p}{(}\PY{n}{input\PYZus{}tensor}\PY{o}{.}\PY{n}{shape}\PY{p}{[}\PY{o}{\PYZhy{}}\PY{l+m+mi}{1}\PY{p}{]}\PY{p}{)}
    \PY{n}{a} \PY{o}{=} \PY{n}{tf}\PY{o}{.}\PY{n}{reshape}\PY{p}{(}\PY{n}{input\PYZus{}tensor}\PY{p}{,} \PY{p}{[}\PY{o}{\PYZhy{}}\PY{l+m+mi}{1}\PY{p}{,} \PY{n}{channels}\PY{p}{]}\PY{p}{)}
    \PY{n}{n} \PY{o}{=} \PY{n}{tf}\PY{o}{.}\PY{n}{shape}\PY{p}{(}\PY{n}{a}\PY{p}{)}\PY{p}{[}\PY{l+m+mi}{0}\PY{p}{]}
    \PY{n}{gram} \PY{o}{=} \PY{n}{tf}\PY{o}{.}\PY{n}{matmul}\PY{p}{(}\PY{n}{a}\PY{p}{,} \PY{n}{a}\PY{p}{,} \PY{n}{transpose\PYZus{}a}\PY{o}{=}\PY{k+kc}{True}\PY{p}{)}
    \PY{k}{return} \PY{n}{gram}\PY{c+c1}{\PYZsh{}/tf.cast(n, tf.float32)}

\PY{k}{def} \PY{n+nf}{get\PYZus{}style\PYZus{}loss}\PY{p}{(}\PY{n}{style}\PY{p}{,} \PY{n}{combination}\PY{p}{)}\PY{p}{:}
    \PY{k}{assert} \PY{n}{K}\PY{o}{.}\PY{n}{ndim}\PY{p}{(}\PY{n}{style}\PY{p}{)} \PY{o}{==} \PY{l+m+mi}{3}
    \PY{k}{assert} \PY{n}{K}\PY{o}{.}\PY{n}{ndim}\PY{p}{(}\PY{n}{combination}\PY{p}{)} \PY{o}{==} \PY{l+m+mi}{3}
    \PY{n}{S} \PY{o}{=} \PY{n}{gram\PYZus{}matrix}\PY{p}{(}\PY{n}{style}\PY{p}{)}
    \PY{n}{C} \PY{o}{=} \PY{n}{gram\PYZus{}matrix}\PY{p}{(}\PY{n}{combination}\PY{p}{)}
    \PY{n}{channels} \PY{o}{=} \PY{l+m+mi}{3}
    \PY{n}{size} \PY{o}{=} \PY{n}{img\PYZus{}nrows}\PY{o}{*}\PY{n}{img\PYZus{}ncols}
    \PY{k}{return} \PY{n}{K}\PY{o}{.}\PY{n}{sum}\PY{p}{(}\PY{n}{K}\PY{o}{.}\PY{n}{square}\PY{p}{(}\PY{n}{S} \PY{o}{\PYZhy{}} \PY{n}{C}\PY{p}{)}\PY{p}{)}\PY{c+c1}{\PYZsh{}/(4.0 * (channels ** 2) * (size ** 2))}
\end{Verbatim}
\end{tcolorbox}

    \begin{tcolorbox}[breakable, size=fbox, boxrule=1pt, pad at break*=1mm,colback=cellbackground, colframe=cellborder]
\prompt{In}{incolor}{220}{\boxspacing}
\begin{Verbatim}[commandchars=\\\{\}]
\PY{n}{content\PYZus{}weight}\PY{o}{=}\PY{l+m+mf}{0.025} 
\PY{n}{style\PYZus{}weight}\PY{o}{=}\PY{l+m+mf}{1.0}
\PY{c+c1}{\PYZsh{} combine these loss functions into a single scalar}
\PY{n}{loss} \PY{o}{=} \PY{n}{K}\PY{o}{.}\PY{n}{variable}\PY{p}{(}\PY{l+m+mf}{0.0}\PY{p}{)}
\PY{n}{layer\PYZus{}features} \PY{o}{=} \PY{n}{outputs\PYZus{}dict}\PY{p}{[}\PY{l+s+s1}{\PYZsq{}}\PY{l+s+s1}{block5\PYZus{}conv2}\PY{l+s+s1}{\PYZsq{}}\PY{p}{]}
\PY{n}{base\PYZus{}image\PYZus{}features} \PY{o}{=} \PY{n}{layer\PYZus{}features}\PY{p}{[}\PY{l+m+mi}{0}\PY{p}{,} \PY{p}{:}\PY{p}{,} \PY{p}{:}\PY{p}{,} \PY{p}{:}\PY{p}{]}
\PY{n}{combination\PYZus{}features} \PY{o}{=} \PY{n}{layer\PYZus{}features}\PY{p}{[}\PY{l+m+mi}{2}\PY{p}{,} \PY{p}{:}\PY{p}{,} \PY{p}{:}\PY{p}{,} \PY{p}{:}\PY{p}{]}
\PY{n+nb}{print}\PY{p}{(}\PY{l+s+s1}{\PYZsq{}}\PY{l+s+s1}{Layer Feature for Content Layers :: }\PY{l+s+s1}{\PYZsq{}}\PY{o}{+}\PY{n+nb}{str}\PY{p}{(}\PY{n}{layer\PYZus{}features}\PY{p}{)}\PY{p}{)}
\PY{n+nb}{print}\PY{p}{(}\PY{l+s+s1}{\PYZsq{}}\PY{l+s+s1}{Base Image Feature :: }\PY{l+s+s1}{\PYZsq{}}\PY{o}{+}\PY{n+nb}{str}\PY{p}{(}\PY{n}{base\PYZus{}image\PYZus{}features}\PY{p}{)}\PY{p}{)}
\PY{n+nb}{print}\PY{p}{(}\PY{l+s+s1}{\PYZsq{}}\PY{l+s+s1}{Combination Image Feature for Content Layers:: }\PY{l+s+s1}{\PYZsq{}}\PY{o}{+}\PY{n+nb}{str}\PY{p}{(}\PY{n}{combination\PYZus{}features}\PY{p}{)}\PY{o}{+}\PY{l+s+s1}{\PYZsq{}}\PY{l+s+se}{\PYZbs{}n}\PY{l+s+s1}{\PYZsq{}}\PY{p}{)}
\PY{n}{loss} \PY{o}{=} \PY{n}{loss} \PY{o}{+} \PY{n}{content\PYZus{}weight} \PY{o}{*} \PY{n}{get\PYZus{}content\PYZus{}loss}\PY{p}{(}\PY{n}{base\PYZus{}image\PYZus{}features}\PY{p}{,}
                                      \PY{n}{combination\PYZus{}features}\PY{p}{)}

\PY{n}{feature\PYZus{}layers} \PY{o}{=} \PY{p}{[}\PY{l+s+s1}{\PYZsq{}}\PY{l+s+s1}{block1\PYZus{}conv1}\PY{l+s+s1}{\PYZsq{}}\PY{p}{,} \PY{l+s+s1}{\PYZsq{}}\PY{l+s+s1}{block2\PYZus{}conv1}\PY{l+s+s1}{\PYZsq{}}\PY{p}{,}
                  \PY{l+s+s1}{\PYZsq{}}\PY{l+s+s1}{block3\PYZus{}conv1}\PY{l+s+s1}{\PYZsq{}}\PY{p}{,} \PY{l+s+s1}{\PYZsq{}}\PY{l+s+s1}{block4\PYZus{}conv1}\PY{l+s+s1}{\PYZsq{}}\PY{p}{,}
                  \PY{l+s+s1}{\PYZsq{}}\PY{l+s+s1}{block5\PYZus{}conv1}\PY{l+s+s1}{\PYZsq{}}\PY{p}{]}
\PY{k}{for} \PY{n}{layer\PYZus{}name} \PY{o+ow}{in} \PY{n}{feature\PYZus{}layers}\PY{p}{:}
    \PY{n}{layer\PYZus{}features} \PY{o}{=} \PY{n}{outputs\PYZus{}dict}\PY{p}{[}\PY{n}{layer\PYZus{}name}\PY{p}{]}
    \PY{n}{style\PYZus{}reference\PYZus{}features} \PY{o}{=} \PY{n}{layer\PYZus{}features}\PY{p}{[}\PY{l+m+mi}{1}\PY{p}{,} \PY{p}{:}\PY{p}{,} \PY{p}{:}\PY{p}{,} \PY{p}{:}\PY{p}{]}
    \PY{n}{combination\PYZus{}features} \PY{o}{=} \PY{n}{layer\PYZus{}features}\PY{p}{[}\PY{l+m+mi}{2}\PY{p}{,} \PY{p}{:}\PY{p}{,} \PY{p}{:}\PY{p}{,} \PY{p}{:}\PY{p}{]}
    \PY{n+nb}{print}\PY{p}{(}\PY{l+s+s1}{\PYZsq{}}\PY{l+s+s1}{Layer Feature for Style Layers :: }\PY{l+s+s1}{\PYZsq{}}\PY{o}{+}\PY{n+nb}{str}\PY{p}{(}\PY{n}{layer\PYZus{}features}\PY{p}{)}\PY{p}{)}
    \PY{n+nb}{print}\PY{p}{(}\PY{l+s+s1}{\PYZsq{}}\PY{l+s+s1}{Style Image Feature :: }\PY{l+s+s1}{\PYZsq{}}\PY{o}{+}\PY{n+nb}{str}\PY{p}{(}\PY{n}{style\PYZus{}reference\PYZus{}features}\PY{p}{)}\PY{p}{)}
    \PY{n+nb}{print}\PY{p}{(}\PY{l+s+s1}{\PYZsq{}}\PY{l+s+s1}{Combination Image Feature for Style Layers:: }\PY{l+s+s1}{\PYZsq{}}\PY{o}{+}\PY{n+nb}{str}\PY{p}{(}\PY{n}{combination\PYZus{}features}\PY{p}{)}\PY{o}{+}\PY{l+s+s1}{\PYZsq{}}\PY{l+s+se}{\PYZbs{}n}\PY{l+s+s1}{\PYZsq{}}\PY{p}{)}
    \PY{n}{sl} \PY{o}{=} \PY{n}{get\PYZus{}style\PYZus{}loss}\PY{p}{(}\PY{n}{style\PYZus{}reference\PYZus{}features}\PY{p}{,} \PY{n}{combination\PYZus{}features}\PY{p}{)}
    \PY{n}{loss} \PY{o}{=} \PY{n}{loss} \PY{o}{+} \PY{p}{(}\PY{n}{style\PYZus{}weight} \PY{o}{/} \PY{n+nb}{len}\PY{p}{(}\PY{n}{feature\PYZus{}layers}\PY{p}{)}\PY{p}{)} \PY{o}{*} \PY{n}{sl}
\end{Verbatim}
\end{tcolorbox}

    \begin{Verbatim}[commandchars=\\\{\}]
Layer Feature for Content Layers :: Tensor("block5\_conv2\_8/Relu:0", shape=(3,
25, 43, 512), dtype=float32)
Base Image Feature :: Tensor("strided\_slice\_176:0", shape=(25, 43, 512),
dtype=float32)
Combination Image Feature for Content Layers:: Tensor("strided\_slice\_177:0",
shape=(25, 43, 512), dtype=float32)

Layer Feature for Style Layers :: Tensor("block1\_conv1\_8/Relu:0", shape=(3, 400,
701, 64), dtype=float32)
Style Image Feature :: Tensor("strided\_slice\_178:0", shape=(400, 701, 64),
dtype=float32)
Combination Image Feature for Style Layers:: Tensor("strided\_slice\_179:0",
shape=(400, 701, 64), dtype=float32)

Layer Feature for Style Layers :: Tensor("block2\_conv1\_8/Relu:0", shape=(3, 200,
350, 128), dtype=float32)
Style Image Feature :: Tensor("strided\_slice\_182:0", shape=(200, 350, 128),
dtype=float32)
Combination Image Feature for Style Layers:: Tensor("strided\_slice\_183:0",
shape=(200, 350, 128), dtype=float32)

Layer Feature for Style Layers :: Tensor("block3\_conv1\_8/Relu:0", shape=(3, 100,
175, 256), dtype=float32)
Style Image Feature :: Tensor("strided\_slice\_186:0", shape=(100, 175, 256),
dtype=float32)
Combination Image Feature for Style Layers:: Tensor("strided\_slice\_187:0",
shape=(100, 175, 256), dtype=float32)

Layer Feature for Style Layers :: Tensor("block4\_conv1\_8/Relu:0", shape=(3, 50,
87, 512), dtype=float32)
Style Image Feature :: Tensor("strided\_slice\_190:0", shape=(50, 87, 512),
dtype=float32)
Combination Image Feature for Style Layers:: Tensor("strided\_slice\_191:0",
shape=(50, 87, 512), dtype=float32)

Layer Feature for Style Layers :: Tensor("block5\_conv1\_8/Relu:0", shape=(3, 25,
43, 512), dtype=float32)
Style Image Feature :: Tensor("strided\_slice\_194:0", shape=(25, 43, 512),
dtype=float32)
Combination Image Feature for Style Layers:: Tensor("strided\_slice\_195:0",
shape=(25, 43, 512), dtype=float32)

    \end{Verbatim}

    \begin{tcolorbox}[breakable, size=fbox, boxrule=1pt, pad at break*=1mm,colback=cellbackground, colframe=cellborder]
\prompt{In}{incolor}{221}{\boxspacing}
\begin{Verbatim}[commandchars=\\\{\}]
\PY{k}{def} \PY{n+nf}{deprocess\PYZus{}image}\PY{p}{(}\PY{n}{x}\PY{p}{)}\PY{p}{:}
    \PY{k}{if} \PY{n}{K}\PY{o}{.}\PY{n}{image\PYZus{}data\PYZus{}format}\PY{p}{(}\PY{p}{)} \PY{o}{==} \PY{l+s+s1}{\PYZsq{}}\PY{l+s+s1}{channels\PYZus{}first}\PY{l+s+s1}{\PYZsq{}}\PY{p}{:}
        \PY{n}{x} \PY{o}{=} \PY{n}{x}\PY{o}{.}\PY{n}{reshape}\PY{p}{(}\PY{p}{(}\PY{l+m+mi}{3}\PY{p}{,} \PY{n}{img\PYZus{}nrows}\PY{p}{,} \PY{n}{img\PYZus{}ncols}\PY{p}{)}\PY{p}{)}
        \PY{n}{x} \PY{o}{=} \PY{n}{x}\PY{o}{.}\PY{n}{transpose}\PY{p}{(}\PY{p}{(}\PY{l+m+mi}{1}\PY{p}{,} \PY{l+m+mi}{2}\PY{p}{,} \PY{l+m+mi}{0}\PY{p}{)}\PY{p}{)}
    \PY{k}{else}\PY{p}{:}
        \PY{n}{x} \PY{o}{=} \PY{n}{x}\PY{o}{.}\PY{n}{reshape}\PY{p}{(}\PY{p}{(}\PY{n}{img\PYZus{}nrows}\PY{p}{,} \PY{n}{img\PYZus{}ncols}\PY{p}{,} \PY{l+m+mi}{3}\PY{p}{)}\PY{p}{)}
    \PY{c+c1}{\PYZsh{} Remove zero\PYZhy{}center by mean pixel}
    \PY{n}{x}\PY{p}{[}\PY{p}{:}\PY{p}{,} \PY{p}{:}\PY{p}{,} \PY{l+m+mi}{0}\PY{p}{]} \PY{o}{+}\PY{o}{=} \PY{l+m+mf}{103.939}
    \PY{n}{x}\PY{p}{[}\PY{p}{:}\PY{p}{,} \PY{p}{:}\PY{p}{,} \PY{l+m+mi}{1}\PY{p}{]} \PY{o}{+}\PY{o}{=} \PY{l+m+mf}{116.779}
    \PY{n}{x}\PY{p}{[}\PY{p}{:}\PY{p}{,} \PY{p}{:}\PY{p}{,} \PY{l+m+mi}{2}\PY{p}{]} \PY{o}{+}\PY{o}{=} \PY{l+m+mf}{123.68}
    \PY{c+c1}{\PYZsh{} \PYZsq{}BGR\PYZsq{}\PYZhy{}\PYZgt{}\PYZsq{}RGB\PYZsq{}}
    \PY{n}{x} \PY{o}{=} \PY{n}{x}\PY{p}{[}\PY{p}{:}\PY{p}{,} \PY{p}{:}\PY{p}{,} \PY{p}{:}\PY{p}{:}\PY{o}{\PYZhy{}}\PY{l+m+mi}{1}\PY{p}{]}
    \PY{n}{x} \PY{o}{=} \PY{n}{np}\PY{o}{.}\PY{n}{clip}\PY{p}{(}\PY{n}{x}\PY{p}{,} \PY{l+m+mi}{0}\PY{p}{,} \PY{l+m+mi}{255}\PY{p}{)}\PY{o}{.}\PY{n}{astype}\PY{p}{(}\PY{l+s+s1}{\PYZsq{}}\PY{l+s+s1}{uint8}\PY{l+s+s1}{\PYZsq{}}\PY{p}{)}
    \PY{k}{return} \PY{n}{x}
\end{Verbatim}
\end{tcolorbox}

    \begin{tcolorbox}[breakable, size=fbox, boxrule=1pt, pad at break*=1mm,colback=cellbackground, colframe=cellborder]
\prompt{In}{incolor}{222}{\boxspacing}
\begin{Verbatim}[commandchars=\\\{\}]
\PY{c+c1}{\PYZsh{} get the gradients of the generated image wrt the loss}
\PY{n}{grads} \PY{o}{=} \PY{n}{K}\PY{o}{.}\PY{n}{gradients}\PY{p}{(}\PY{n}{loss}\PY{p}{,} \PY{n}{combination\PYZus{}image}\PY{p}{)}
\PY{n}{grads}
\end{Verbatim}
\end{tcolorbox}

            \begin{tcolorbox}[breakable, size=fbox, boxrule=.5pt, pad at break*=1mm, opacityfill=0]
\prompt{Out}{outcolor}{222}{\boxspacing}
\begin{Verbatim}[commandchars=\\\{\}]
[<tf.Tensor 'gradients\_8/concat\_8\_grad/Slice\_2:0' shape=(1, 400, 701, 3)
dtype=float32>]
\end{Verbatim}
\end{tcolorbox}
        
    \begin{tcolorbox}[breakable, size=fbox, boxrule=1pt, pad at break*=1mm,colback=cellbackground, colframe=cellborder]
\prompt{In}{incolor}{223}{\boxspacing}
\begin{Verbatim}[commandchars=\\\{\}]
\PY{n}{outputs} \PY{o}{=} \PY{p}{[}\PY{n}{loss}\PY{p}{]}
\PY{k}{if} \PY{n+nb}{isinstance}\PY{p}{(}\PY{n}{grads}\PY{p}{,} \PY{p}{(}\PY{n+nb}{list}\PY{p}{,}\PY{n+nb}{tuple}\PY{p}{)}\PY{p}{)}\PY{p}{:}
    \PY{n}{outputs} \PY{o}{+}\PY{o}{=} \PY{n}{grads}
\PY{k}{else}\PY{p}{:}
    \PY{n}{outputs}\PY{o}{.}\PY{n}{append}\PY{p}{(}\PY{n}{grads}\PY{p}{)}
\PY{n}{f\PYZus{}outputs} \PY{o}{=} \PY{n}{K}\PY{o}{.}\PY{n}{function}\PY{p}{(}\PY{p}{[}\PY{n}{combination\PYZus{}image}\PY{p}{]}\PY{p}{,} \PY{n}{outputs}\PY{p}{)}
\PY{n}{f\PYZus{}outputs}
\end{Verbatim}
\end{tcolorbox}

            \begin{tcolorbox}[breakable, size=fbox, boxrule=.5pt, pad at break*=1mm, opacityfill=0]
\prompt{Out}{outcolor}{223}{\boxspacing}
\begin{Verbatim}[commandchars=\\\{\}]
<keras.backend.GraphExecutionFunction at 0x2c4cbefd0>
\end{Verbatim}
\end{tcolorbox}
        
    \begin{tcolorbox}[breakable, size=fbox, boxrule=1pt, pad at break*=1mm,colback=cellbackground, colframe=cellborder]
\prompt{In}{incolor}{224}{\boxspacing}
\begin{Verbatim}[commandchars=\\\{\}]
\PY{c+c1}{\PYZsh{} run scipy\PYZhy{}based optimization (L\PYZhy{}BFGS) over the pixels of the generated image}
\PY{c+c1}{\PYZsh{} so as to minimize the neural style loss}
\PY{n}{x\PYZus{}opt} \PY{o}{=} \PY{n}{preprocess\PYZus{}image}\PY{p}{(}\PY{n}{base\PYZus{}image\PYZus{}path}\PY{p}{)}
\end{Verbatim}
\end{tcolorbox}

    \begin{tcolorbox}[breakable, size=fbox, boxrule=1pt, pad at break*=1mm,colback=cellbackground, colframe=cellborder]
\prompt{In}{incolor}{225}{\boxspacing}
\begin{Verbatim}[commandchars=\\\{\}]
\PY{k}{def} \PY{n+nf}{eval\PYZus{}loss\PYZus{}and\PYZus{}grads}\PY{p}{(}\PY{n}{x}\PY{p}{)}\PY{p}{:}
    \PY{k}{if} \PY{n}{K}\PY{o}{.}\PY{n}{image\PYZus{}data\PYZus{}format}\PY{p}{(}\PY{p}{)} \PY{o}{==} \PY{l+s+s1}{\PYZsq{}}\PY{l+s+s1}{channels\PYZus{}first}\PY{l+s+s1}{\PYZsq{}}\PY{p}{:}
        \PY{n}{x} \PY{o}{=} \PY{n}{x}\PY{o}{.}\PY{n}{reshape}\PY{p}{(}\PY{p}{(}\PY{l+m+mi}{1}\PY{p}{,} \PY{l+m+mi}{3}\PY{p}{,} \PY{n}{img\PYZus{}nrows}\PY{p}{,} \PY{n}{img\PYZus{}ncols}\PY{p}{)}\PY{p}{)}
    \PY{k}{else}\PY{p}{:}
        \PY{n}{x} \PY{o}{=} \PY{n}{x}\PY{o}{.}\PY{n}{reshape}\PY{p}{(}\PY{p}{(}\PY{l+m+mi}{1}\PY{p}{,} \PY{n}{img\PYZus{}nrows}\PY{p}{,} \PY{n}{img\PYZus{}ncols}\PY{p}{,} \PY{l+m+mi}{3}\PY{p}{)}\PY{p}{)}
    \PY{n}{outs} \PY{o}{=} \PY{n}{f\PYZus{}outputs}\PY{p}{(}\PY{p}{[}\PY{n}{x}\PY{p}{]}\PY{p}{)}
    \PY{n}{loss\PYZus{}value} \PY{o}{=} \PY{n}{outs}\PY{p}{[}\PY{l+m+mi}{0}\PY{p}{]}
    \PY{k}{if} \PY{n+nb}{len}\PY{p}{(}\PY{n}{outs}\PY{p}{[}\PY{l+m+mi}{1}\PY{p}{:}\PY{p}{]}\PY{p}{)} \PY{o}{==} \PY{l+m+mi}{1}\PY{p}{:}
        \PY{n}{grad\PYZus{}values} \PY{o}{=} \PY{n}{outs}\PY{p}{[}\PY{l+m+mi}{1}\PY{p}{]}\PY{o}{.}\PY{n}{flatten}\PY{p}{(}\PY{p}{)}\PY{o}{.}\PY{n}{astype}\PY{p}{(}\PY{l+s+s1}{\PYZsq{}}\PY{l+s+s1}{float64}\PY{l+s+s1}{\PYZsq{}}\PY{p}{)}
    \PY{k}{else}\PY{p}{:}
        \PY{n}{grad\PYZus{}values} \PY{o}{=} \PY{n}{np}\PY{o}{.}\PY{n}{array}\PY{p}{(}\PY{n}{outs}\PY{p}{[}\PY{l+m+mi}{1}\PY{p}{:}\PY{p}{]}\PY{p}{)}\PY{o}{.}\PY{n}{flatten}\PY{p}{(}\PY{p}{)}\PY{o}{.}\PY{n}{astype}\PY{p}{(}\PY{l+s+s1}{\PYZsq{}}\PY{l+s+s1}{float64}\PY{l+s+s1}{\PYZsq{}}\PY{p}{)}
    \PY{k}{return} \PY{n}{loss\PYZus{}value}\PY{p}{,} \PY{n}{grad\PYZus{}values}
\end{Verbatim}
\end{tcolorbox}

    \begin{tcolorbox}[breakable, size=fbox, boxrule=1pt, pad at break*=1mm,colback=cellbackground, colframe=cellborder]
\prompt{In}{incolor}{226}{\boxspacing}
\begin{Verbatim}[commandchars=\\\{\}]
\PY{k}{class} \PY{n+nc}{Evaluator}\PY{p}{(}\PY{n+nb}{object}\PY{p}{)}\PY{p}{:}

    \PY{k}{def} \PY{n+nf+fm}{\PYZus{}\PYZus{}init\PYZus{}\PYZus{}}\PY{p}{(}\PY{n+nb+bp}{self}\PY{p}{)}\PY{p}{:}
        \PY{n+nb+bp}{self}\PY{o}{.}\PY{n}{loss\PYZus{}value} \PY{o}{=} \PY{k+kc}{None}
        \PY{n+nb+bp}{self}\PY{o}{.}\PY{n}{grads\PYZus{}values} \PY{o}{=} \PY{k+kc}{None}

    \PY{k}{def} \PY{n+nf}{loss}\PY{p}{(}\PY{n+nb+bp}{self}\PY{p}{,} \PY{n}{x}\PY{p}{)}\PY{p}{:}
        \PY{k}{assert} \PY{n+nb+bp}{self}\PY{o}{.}\PY{n}{loss\PYZus{}value} \PY{o+ow}{is} \PY{k+kc}{None}
        \PY{n}{loss\PYZus{}value}\PY{p}{,} \PY{n}{grad\PYZus{}values} \PY{o}{=} \PY{n}{eval\PYZus{}loss\PYZus{}and\PYZus{}grads}\PY{p}{(}\PY{n}{x}\PY{p}{)}
        \PY{n+nb+bp}{self}\PY{o}{.}\PY{n}{loss\PYZus{}value} \PY{o}{=} \PY{n}{loss\PYZus{}value}
        \PY{n+nb+bp}{self}\PY{o}{.}\PY{n}{grad\PYZus{}values} \PY{o}{=} \PY{n}{grad\PYZus{}values}
        \PY{k}{return} \PY{n+nb+bp}{self}\PY{o}{.}\PY{n}{loss\PYZus{}value}

    \PY{k}{def} \PY{n+nf}{grads}\PY{p}{(}\PY{n+nb+bp}{self}\PY{p}{,} \PY{n}{x}\PY{p}{)}\PY{p}{:}
        \PY{k}{assert} \PY{n+nb+bp}{self}\PY{o}{.}\PY{n}{loss\PYZus{}value} \PY{o+ow}{is} \PY{o+ow}{not} \PY{k+kc}{None}
        \PY{n}{grad\PYZus{}values} \PY{o}{=} \PY{n}{np}\PY{o}{.}\PY{n}{copy}\PY{p}{(}\PY{n+nb+bp}{self}\PY{o}{.}\PY{n}{grad\PYZus{}values}\PY{p}{)}
        \PY{n+nb+bp}{self}\PY{o}{.}\PY{n}{loss\PYZus{}value} \PY{o}{=} \PY{k+kc}{None}
        \PY{n+nb+bp}{self}\PY{o}{.}\PY{n}{grad\PYZus{}values} \PY{o}{=} \PY{k+kc}{None}
        \PY{k}{return} \PY{n}{grad\PYZus{}values}
\end{Verbatim}
\end{tcolorbox}

    \begin{tcolorbox}[breakable, size=fbox, boxrule=1pt, pad at break*=1mm,colback=cellbackground, colframe=cellborder]
\prompt{In}{incolor}{227}{\boxspacing}
\begin{Verbatim}[commandchars=\\\{\}]
\PY{n}{evaluator} \PY{o}{=} \PY{n}{Evaluator}\PY{p}{(}\PY{p}{)}
\end{Verbatim}
\end{tcolorbox}

    \begin{tcolorbox}[breakable, size=fbox, boxrule=1pt, pad at break*=1mm,colback=cellbackground, colframe=cellborder]
\prompt{In}{incolor}{228}{\boxspacing}
\begin{Verbatim}[commandchars=\\\{\}]
\PY{n}{iterations}\PY{o}{=}\PY{l+m+mi}{20}
\PY{c+c1}{\PYZsh{} Store our best result}
\PY{n}{best\PYZus{}loss}\PY{p}{,} \PY{n}{best\PYZus{}img} \PY{o}{=} \PY{n+nb}{float}\PY{p}{(}\PY{l+s+s1}{\PYZsq{}}\PY{l+s+s1}{inf}\PY{l+s+s1}{\PYZsq{}}\PY{p}{)}\PY{p}{,} \PY{k+kc}{None}
\PY{k}{for} \PY{n}{i} \PY{o+ow}{in} \PY{n+nb}{range}\PY{p}{(}\PY{n}{iterations}\PY{p}{)}\PY{p}{:}
    \PY{n+nb}{print}\PY{p}{(}\PY{l+s+s1}{\PYZsq{}}\PY{l+s+s1}{Start of iteration}\PY{l+s+s1}{\PYZsq{}}\PY{p}{,} \PY{n}{i}\PY{p}{)}
    \PY{n}{x\PYZus{}opt}\PY{p}{,} \PY{n}{min\PYZus{}val}\PY{p}{,} \PY{n}{info}\PY{o}{=} \PY{n}{fmin\PYZus{}l\PYZus{}bfgs\PYZus{}b}\PY{p}{(}\PY{n}{evaluator}\PY{o}{.}\PY{n}{loss}\PY{p}{,} 
                                        \PY{n}{x\PYZus{}opt}\PY{o}{.}\PY{n}{flatten}\PY{p}{(}\PY{p}{)}\PY{p}{,} 
                                        \PY{n}{fprime}\PY{o}{=}\PY{n}{evaluator}\PY{o}{.}\PY{n}{grads}\PY{p}{,}
                                        \PY{n}{maxfun}\PY{o}{=}\PY{l+m+mi}{20}\PY{p}{,}
                                        \PY{n}{disp}\PY{o}{=}\PY{k+kc}{True}\PY{p}{,}
                                       \PY{p}{)}
    \PY{n+nb}{print}\PY{p}{(}\PY{l+s+s1}{\PYZsq{}}\PY{l+s+s1}{Current loss value:}\PY{l+s+s1}{\PYZsq{}}\PY{p}{,} \PY{n}{min\PYZus{}val}\PY{p}{)}
    \PY{k}{if} \PY{n}{min\PYZus{}val} \PY{o}{\PYZlt{}} \PY{n}{best\PYZus{}loss}\PY{p}{:}
        \PY{c+c1}{\PYZsh{} Update best loss and best image from total loss. }
        \PY{n}{best\PYZus{}loss} \PY{o}{=} \PY{n}{min\PYZus{}val}
        \PY{n}{best\PYZus{}img} \PY{o}{=} \PY{n}{x\PYZus{}opt}\PY{o}{.}\PY{n}{copy}\PY{p}{(}\PY{p}{)}
\end{Verbatim}
\end{tcolorbox}

    \begin{Verbatim}[commandchars=\\\{\}]
Start of iteration 0
    \end{Verbatim}

    \begin{Verbatim}[commandchars=\\\{\}]
2023-06-16 14:56:08.327757: W tensorflow/c/c\_api.cc:300] Operation
'\{name:'Variable\_26/Assign' id:8444 op device:\{requested: '', assigned: ''\}
def:\{\{\{node Variable\_26/Assign\}\} =
AssignVariableOp[\_has\_manual\_control\_dependencies=true, dtype=DT\_FLOAT,
validate\_shape=false](Variable\_26, Variable\_26/Initializer/initial\_value)\}\}' was
changed by setting attribute after it was run by a session. This mutation will
have no effect, and will trigger an error in the future. Either don't modify
nodes after running them or create a new session.
    \end{Verbatim}

    \begin{Verbatim}[commandchars=\\\{\}]
RUNNING THE L-BFGS-B CODE

           * * *

Machine precision = 2.220D-16
 N =       841200     M =           10

At X0         0 variables are exactly at the bounds

At iterate    0    f=  1.28266D+23    |proj g|=  4.37175D+17
    \end{Verbatim}

    \begin{Verbatim}[commandchars=\\\{\}]
 This problem is unconstrained.
    \end{Verbatim}

    \begin{Verbatim}[commandchars=\\\{\}]

At iterate    1    f=  1.16952D+23    |proj g|=  4.29219D+17

At iterate    2    f=  7.58508D+22    |proj g|=  5.43244D+17

At iterate    3    f=  4.09127D+22    |proj g|=  3.17688D+17

At iterate    4    f=  2.05052D+22    |proj g|=  2.71112D+17

At iterate    5    f=  1.18706D+22    |proj g|=  1.07781D+17

At iterate    6    f=  9.11366D+21    |proj g|=  1.78400D+17

At iterate    7    f=  6.74590D+21    |proj g|=  7.53673D+16

At iterate    8    f=  5.57402D+21    |proj g|=  5.11874D+16

At iterate    9    f=  4.46768D+21    |proj g|=  3.63965D+16

At iterate   10    f=  3.77621D+21    |proj g|=  1.25330D+17

At iterate   11    f=  2.88212D+21    |proj g|=  2.73147D+16

At iterate   12    f=  2.65678D+21    |proj g|=  1.48898D+16

At iterate   13    f=  2.22423D+21    |proj g|=  1.65953D+16

At iterate   14    f=  1.78377D+21    |proj g|=  2.56928D+16

At iterate   15    f=  1.47444D+21    |proj g|=  1.58641D+16

At iterate   16    f=  1.35859D+21    |proj g|=  2.06004D+16

           * * *

Tit   = total number of iterations
Tnf   = total number of function evaluations
Tnint = total number of segments explored during Cauchy searches
Skip  = number of BFGS updates skipped
Nact  = number of active bounds at final generalized Cauchy point
Projg = norm of the final projected gradient
F     = final function value

           * * *

   N    Tit     Tnf  Tnint  Skip  Nact     Projg        F
*****     16     21      1     0     0   2.060D+16   1.359D+21
  F =   1.3585940034519281E+021

STOP: TOTAL NO. of f AND g EVALUATIONS EXCEEDS LIMIT
Current loss value: 1.358594e+21
Start of iteration 1
RUNNING THE L-BFGS-B CODE

           * * *

Machine precision = 2.220D-16
 N =       841200     M =           10

At X0         0 variables are exactly at the bounds

At iterate    0    f=  1.35859D+21    |proj g|=  2.06004D+16
    \end{Verbatim}

    \begin{Verbatim}[commandchars=\\\{\}]
 This problem is unconstrained.
    \end{Verbatim}

    \begin{Verbatim}[commandchars=\\\{\}]

At iterate    1    f=  1.33224D+21    |proj g|=  1.72262D+16

At iterate    2    f=  1.20902D+21    |proj g|=  6.46567D+15

At iterate    3    f=  1.15669D+21    |proj g|=  6.27291D+15

At iterate    4    f=  1.03769D+21    |proj g|=  1.48744D+16

At iterate    5    f=  9.74922D+20    |proj g|=  1.10461D+16

At iterate    6    f=  9.32556D+20    |proj g|=  7.69400D+15

At iterate    7    f=  8.20241D+20    |proj g|=  5.49688D+15

At iterate    8    f=  7.55906D+20    |proj g|=  6.71566D+15

At iterate    9    f=  7.10689D+20    |proj g|=  1.41905D+16

At iterate   10    f=  6.50206D+20    |proj g|=  4.54267D+15

At iterate   11    f=  6.22463D+20    |proj g|=  1.08975D+16

At iterate   12    f=  5.74963D+20    |proj g|=  6.62106D+15

At iterate   13    f=  5.27467D+20    |proj g|=  6.76265D+15

At iterate   14    f=  4.89117D+20    |proj g|=  3.79493D+15

At iterate   15    f=  4.76924D+20    |proj g|=  7.77426D+15

At iterate   16    f=  4.58533D+20    |proj g|=  3.44620D+15

At iterate   17    f=  4.41064D+20    |proj g|=  3.67949D+15

At iterate   18    f=  4.14127D+20    |proj g|=  6.14525D+15

           * * *

Tit   = total number of iterations
Tnf   = total number of function evaluations
Tnint = total number of segments explored during Cauchy searches
Skip  = number of BFGS updates skipped
Nact  = number of active bounds at final generalized Cauchy point
Projg = norm of the final projected gradient
F     = final function value

           * * *

   N    Tit     Tnf  Tnint  Skip  Nact     Projg        F
*****     18     21      1     0     0   6.145D+15   4.141D+20
  F =   4.1412674451624952E+020

STOP: TOTAL NO. of f AND g EVALUATIONS EXCEEDS LIMIT
Current loss value: 4.1412674e+20
Start of iteration 2
RUNNING THE L-BFGS-B CODE

           * * *

Machine precision = 2.220D-16
 N =       841200     M =           10

At X0         0 variables are exactly at the bounds

At iterate    0    f=  4.14127D+20    |proj g|=  6.14525D+15
    \end{Verbatim}

    \begin{Verbatim}[commandchars=\\\{\}]
 This problem is unconstrained.
    \end{Verbatim}

    \begin{Verbatim}[commandchars=\\\{\}]

At iterate    1    f=  4.12410D+20    |proj g|=  5.07414D+15

At iterate    2    f=  4.03537D+20    |proj g|=  2.44056D+15

At iterate    3    f=  3.98831D+20    |proj g|=  3.40373D+15

At iterate    4    f=  3.85870D+20    |proj g|=  4.34621D+15

At iterate    5    f=  3.82713D+20    |proj g|=  8.26096D+15

At iterate    6    f=  3.69165D+20    |proj g|=  2.89373D+15

At iterate    7    f=  3.63237D+20    |proj g|=  2.08066D+15

At iterate    8    f=  3.54669D+20    |proj g|=  2.31852D+15

At iterate    9    f=  3.44734D+20    |proj g|=  2.97622D+15

At iterate   10    f=  3.33110D+20    |proj g|=  3.80135D+15

At iterate   11    f=  3.17553D+20    |proj g|=  3.30910D+15

At iterate   12    f=  3.03337D+20    |proj g|=  6.00335D+15

At iterate   13    f=  2.87314D+20    |proj g|=  3.41562D+15

At iterate   14    f=  2.81902D+20    |proj g|=  2.49338D+15

At iterate   15    f=  2.76089D+20    |proj g|=  3.06568D+15

At iterate   16    f=  2.66803D+20    |proj g|=  3.80967D+15

At iterate   17    f=  2.54731D+20    |proj g|=  3.76587D+15

At iterate   18    f=  2.39215D+20    |proj g|=  3.60764D+15

At iterate   19    f=  2.34492D+20    |proj g|=  7.39750D+15

           * * *

Tit   = total number of iterations
Tnf   = total number of function evaluations
Tnint = total number of segments explored during Cauchy searches
Skip  = number of BFGS updates skipped
Nact  = number of active bounds at final generalized Cauchy point
Projg = norm of the final projected gradient
F     = final function value

           * * *

   N    Tit     Tnf  Tnint  Skip  Nact     Projg        F
*****     19     21      1     0     0   7.397D+15   2.345D+20
  F =   2.3449185969336903E+020

STOP: TOTAL NO. of f AND g EVALUATIONS EXCEEDS LIMIT
Current loss value: 2.3449186e+20
Start of iteration 3
RUNNING THE L-BFGS-B CODE

           * * *

Machine precision = 2.220D-16
 N =       841200     M =           10

At X0         0 variables are exactly at the bounds

At iterate    0    f=  2.34492D+20    |proj g|=  7.39750D+15
    \end{Verbatim}

    \begin{Verbatim}[commandchars=\\\{\}]
 This problem is unconstrained.
    \end{Verbatim}

    \begin{Verbatim}[commandchars=\\\{\}]

At iterate    1    f=  2.25190D+20    |proj g|=  5.27323D+15

At iterate    2    f=  2.14770D+20    |proj g|=  2.91554D+15

At iterate    3    f=  2.11904D+20    |proj g|=  1.69171D+15

At iterate    4    f=  2.07531D+20    |proj g|=  2.29421D+15

At iterate    5    f=  2.05687D+20    |proj g|=  6.24965D+15

At iterate    6    f=  2.00389D+20    |proj g|=  1.76700D+15

At iterate    7    f=  1.99259D+20    |proj g|=  9.05864D+14

At iterate    8    f=  1.96783D+20    |proj g|=  1.93842D+15

At iterate    9    f=  1.93767D+20    |proj g|=  2.41149D+15

At iterate   10    f=  1.90477D+20    |proj g|=  1.74273D+15

At iterate   11    f=  1.87351D+20    |proj g|=  1.28152D+15

At iterate   12    f=  1.84103D+20    |proj g|=  1.38705D+15

At iterate   13    f=  1.79964D+20    |proj g|=  2.79270D+15

At iterate   14    f=  1.74810D+20    |proj g|=  2.97856D+15

At iterate   15    f=  1.71187D+20    |proj g|=  1.99316D+15

At iterate   16    f=  1.67739D+20    |proj g|=  1.43641D+15

At iterate   17    f=  1.63976D+20    |proj g|=  2.67075D+15

At iterate   18    f=  1.59996D+20    |proj g|=  2.15522D+15

           * * *

Tit   = total number of iterations
Tnf   = total number of function evaluations
Tnint = total number of segments explored during Cauchy searches
Skip  = number of BFGS updates skipped
Nact  = number of active bounds at final generalized Cauchy point
Projg = norm of the final projected gradient
F     = final function value

           * * *

   N    Tit     Tnf  Tnint  Skip  Nact     Projg        F
*****     18     21      1     0     0   2.155D+15   1.600D+20
  F =   1.5999577756345251E+020

STOP: TOTAL NO. of f AND g EVALUATIONS EXCEEDS LIMIT
Current loss value: 1.5999578e+20
Start of iteration 4
    \end{Verbatim}

    \begin{Verbatim}[commandchars=\\\{\}]
 This problem is unconstrained.
    \end{Verbatim}

    \begin{Verbatim}[commandchars=\\\{\}]
RUNNING THE L-BFGS-B CODE

           * * *

Machine precision = 2.220D-16
 N =       841200     M =           10

At X0         0 variables are exactly at the bounds

At iterate    0    f=  1.59996D+20    |proj g|=  2.15522D+15

At iterate    1    f=  1.59258D+20    |proj g|=  1.83965D+15

At iterate    2    f=  1.57141D+20    |proj g|=  1.67365D+15

At iterate    3    f=  1.55529D+20    |proj g|=  2.10227D+15

At iterate    4    f=  1.53301D+20    |proj g|=  1.90537D+15

At iterate    5    f=  1.51805D+20    |proj g|=  2.29533D+15

At iterate    6    f=  1.50089D+20    |proj g|=  1.00766D+15

At iterate    7    f=  1.48568D+20    |proj g|=  1.15173D+15

At iterate    8    f=  1.47111D+20    |proj g|=  1.09660D+15

At iterate    9    f=  1.45034D+20    |proj g|=  1.56042D+15

At iterate   10    f=  1.43031D+20    |proj g|=  1.77084D+15

At iterate   11    f=  1.40957D+20    |proj g|=  1.12917D+15

At iterate   12    f=  1.38113D+20    |proj g|=  1.62870D+15

At iterate   13    f=  1.36158D+20    |proj g|=  1.48991D+15

At iterate   14    f=  1.34027D+20    |proj g|=  1.26788D+15

At iterate   15    f=  1.32356D+20    |proj g|=  1.29133D+15

At iterate   16    f=  1.30866D+20    |proj g|=  1.01276D+15

At iterate   17    f=  1.28297D+20    |proj g|=  1.30158D+15

At iterate   18    f=  1.25221D+20    |proj g|=  2.92424D+15

At iterate   19    f=  1.21075D+20    |proj g|=  2.29619D+15

           * * *

Tit   = total number of iterations
Tnf   = total number of function evaluations
Tnint = total number of segments explored during Cauchy searches
Skip  = number of BFGS updates skipped
Nact  = number of active bounds at final generalized Cauchy point
Projg = norm of the final projected gradient
F     = final function value

           * * *

   N    Tit     Tnf  Tnint  Skip  Nact     Projg        F
*****     19     21      1     0     0   2.296D+15   1.211D+20
  F =   1.2107467562520517E+020

STOP: TOTAL NO. of f AND g EVALUATIONS EXCEEDS LIMIT
Current loss value: 1.2107468e+20
Start of iteration 5
RUNNING THE L-BFGS-B CODE

           * * *

Machine precision = 2.220D-16
 N =       841200     M =           10

At X0         0 variables are exactly at the bounds

At iterate    0    f=  1.21075D+20    |proj g|=  2.29619D+15
    \end{Verbatim}

    \begin{Verbatim}[commandchars=\\\{\}]
 This problem is unconstrained.
    \end{Verbatim}

    \begin{Verbatim}[commandchars=\\\{\}]

At iterate    1    f=  1.20397D+20    |proj g|=  1.94188D+15

At iterate    2    f=  1.17400D+20    |proj g|=  1.42639D+15

At iterate    3    f=  1.15349D+20    |proj g|=  1.58451D+15

At iterate    4    f=  1.14439D+20    |proj g|=  1.82519D+15

At iterate    5    f=  1.13801D+20    |proj g|=  8.69226D+14

At iterate    6    f=  1.13186D+20    |proj g|=  7.97898D+14

At iterate    7    f=  1.12156D+20    |proj g|=  1.51936D+15

At iterate    8    f=  1.11289D+20    |proj g|=  1.84693D+15

At iterate    9    f=  1.10197D+20    |proj g|=  9.74588D+14

At iterate   10    f=  1.09417D+20    |proj g|=  7.71134D+14

At iterate   11    f=  1.08552D+20    |proj g|=  1.03753D+15

At iterate   12    f=  1.07154D+20    |proj g|=  1.10336D+15

At iterate   13    f=  1.05145D+20    |proj g|=  1.81567D+15

At iterate   14    f=  1.03490D+20    |proj g|=  1.77907D+15

At iterate   15    f=  1.02281D+20    |proj g|=  8.00154D+14

At iterate   16    f=  1.01559D+20    |proj g|=  8.81630D+14

At iterate   17    f=  1.00884D+20    |proj g|=  8.29503D+14

At iterate   18    f=  9.97435D+19    |proj g|=  7.45461D+14

           * * *

Tit   = total number of iterations
Tnf   = total number of function evaluations
Tnint = total number of segments explored during Cauchy searches
Skip  = number of BFGS updates skipped
Nact  = number of active bounds at final generalized Cauchy point
Projg = norm of the final projected gradient
F     = final function value

           * * *

   N    Tit     Tnf  Tnint  Skip  Nact     Projg        F
*****     18     21      1     0     0   7.455D+14   9.974D+19
  F =   9.9743490339374105E+019

STOP: TOTAL NO. of f AND g EVALUATIONS EXCEEDS LIMIT
Current loss value: 9.974349e+19
Start of iteration 6
RUNNING THE L-BFGS-B CODE

           * * *

Machine precision = 2.220D-16
 N =       841200     M =           10

At X0         0 variables are exactly at the bounds

At iterate    0    f=  9.97435D+19    |proj g|=  7.45461D+14
    \end{Verbatim}

    \begin{Verbatim}[commandchars=\\\{\}]
 This problem is unconstrained.
    \end{Verbatim}

    \begin{Verbatim}[commandchars=\\\{\}]

At iterate    1    f=  9.94413D+19    |proj g|=  5.93626D+14

At iterate    2    f=  9.82400D+19    |proj g|=  9.42657D+14

At iterate    3    f=  9.78808D+19    |proj g|=  1.01938D+15

At iterate    4    f=  9.75242D+19    |proj g|=  6.94788D+14

At iterate    5    f=  9.69498D+19    |proj g|=  7.93938D+14

At iterate    6    f=  9.63020D+19    |proj g|=  1.13287D+15

At iterate    7    f=  9.54519D+19    |proj g|=  1.57486D+15

At iterate    8    f=  9.44823D+19    |proj g|=  9.11736D+14

At iterate    9    f=  9.36166D+19    |proj g|=  1.01786D+15

At iterate   10    f=  9.27148D+19    |proj g|=  1.15722D+15

At iterate   11    f=  9.16242D+19    |proj g|=  1.02215D+15

At iterate   12    f=  9.10752D+19    |proj g|=  1.35522D+15

At iterate   13    f=  9.01609D+19    |proj g|=  7.55887D+14

At iterate   14    f=  8.93465D+19    |proj g|=  8.01316D+14

At iterate   15    f=  8.84354D+19    |proj g|=  1.16558D+15

At iterate   16    f=  8.75514D+19    |proj g|=  8.20058D+14

At iterate   17    f=  8.64621D+19    |proj g|=  1.08642D+15

           * * *

Tit   = total number of iterations
Tnf   = total number of function evaluations
Tnint = total number of segments explored during Cauchy searches
Skip  = number of BFGS updates skipped
Nact  = number of active bounds at final generalized Cauchy point
Projg = norm of the final projected gradient
F     = final function value

           * * *

   N    Tit     Tnf  Tnint  Skip  Nact     Projg        F
*****     17     21      1     0     0   1.086D+15   8.646D+19
  F =   8.6462111155467846E+019

STOP: TOTAL NO. of f AND g EVALUATIONS EXCEEDS LIMIT
Current loss value: 8.646211e+19
Start of iteration 7
RUNNING THE L-BFGS-B CODE

           * * *

Machine precision = 2.220D-16
 N =       841200     M =           10

At X0         0 variables are exactly at the bounds

At iterate    0    f=  8.64621D+19    |proj g|=  1.08642D+15
    \end{Verbatim}

    \begin{Verbatim}[commandchars=\\\{\}]
 This problem is unconstrained.
    \end{Verbatim}

    \begin{Verbatim}[commandchars=\\\{\}]

At iterate    1    f=  8.61700D+19    |proj g|=  6.66139D+14

At iterate    2    f=  8.54444D+19    |proj g|=  7.61263D+14

At iterate    3    f=  8.47085D+19    |proj g|=  1.22982D+15

At iterate    4    f=  8.42733D+19    |proj g|=  1.16035D+15

At iterate    5    f=  8.39877D+19    |proj g|=  6.79927D+14

At iterate    6    f=  8.36360D+19    |proj g|=  6.00692D+14

At iterate    7    f=  8.31310D+19    |proj g|=  8.04422D+14

At iterate    8    f=  8.24519D+19    |proj g|=  1.13216D+15

At iterate    9    f=  8.23040D+19    |proj g|=  2.05927D+15

At iterate   10    f=  8.15263D+19    |proj g|=  6.01157D+14

At iterate   11    f=  8.11941D+19    |proj g|=  5.74650D+14

At iterate   12    f=  8.06388D+19    |proj g|=  8.79165D+14

At iterate   13    f=  7.98610D+19    |proj g|=  1.01203D+15

At iterate   14    f=  7.95468D+19    |proj g|=  1.98416D+15

At iterate   15    f=  7.80585D+19    |proj g|=  1.21694D+15

At iterate   16    f=  7.76627D+19    |proj g|=  4.46106D+14

At iterate   17    f=  7.73238D+19    |proj g|=  5.44133D+14

At iterate   18    f=  7.68531D+19    |proj g|=  5.93565D+14

At iterate   19    f=  7.56466D+19    |proj g|=  9.86058D+14

           * * *

Tit   = total number of iterations
Tnf   = total number of function evaluations
Tnint = total number of segments explored during Cauchy searches
Skip  = number of BFGS updates skipped
Nact  = number of active bounds at final generalized Cauchy point
Projg = norm of the final projected gradient
F     = final function value

           * * *

   N    Tit     Tnf  Tnint  Skip  Nact     Projg        F
*****     19     21      1     0     0   9.861D+14   7.565D+19
  F =   7.5646619893314355E+019

STOP: TOTAL NO. of f AND g EVALUATIONS EXCEEDS LIMIT
Current loss value: 7.564662e+19
Start of iteration 8
RUNNING THE L-BFGS-B CODE

           * * *

Machine precision = 2.220D-16
 N =       841200     M =           10

At X0         0 variables are exactly at the bounds

At iterate    0    f=  7.56466D+19    |proj g|=  9.86058D+14
    \end{Verbatim}

    \begin{Verbatim}[commandchars=\\\{\}]
 This problem is unconstrained.
    \end{Verbatim}

    \begin{Verbatim}[commandchars=\\\{\}]

At iterate    1    f=  7.52715D+19    |proj g|=  7.90328D+14

At iterate    2    f=  7.47013D+19    |proj g|=  8.54676D+14

At iterate    3    f=  7.39951D+19    |proj g|=  8.56168D+14

At iterate    4    f=  7.39166D+19    |proj g|=  1.99861D+15

At iterate    5    f=  7.33367D+19    |proj g|=  7.46427D+14

At iterate    6    f=  7.31604D+19    |proj g|=  4.97835D+14

At iterate    7    f=  7.26959D+19    |proj g|=  6.73322D+14

At iterate    8    f=  7.21723D+19    |proj g|=  8.33296D+14

At iterate    9    f=  7.11803D+19    |proj g|=  8.39466D+14

At iterate   10    f=  7.08345D+19    |proj g|=  1.02394D+15

At iterate   11    f=  7.02667D+19    |proj g|=  5.34105D+14

At iterate   12    f=  6.99951D+19    |proj g|=  6.98794D+14

At iterate   13    f=  6.95736D+19    |proj g|=  6.85612D+14

At iterate   14    f=  6.90121D+19    |proj g|=  6.78423D+14

At iterate   15    f=  6.85168D+19    |proj g|=  1.11006D+15

At iterate   16    f=  6.81679D+19    |proj g|=  5.65731D+14

At iterate   17    f=  6.80180D+19    |proj g|=  4.31638D+14

At iterate   18    f=  6.76439D+19    |proj g|=  4.59754D+14

           * * *

Tit   = total number of iterations
Tnf   = total number of function evaluations
Tnint = total number of segments explored during Cauchy searches
Skip  = number of BFGS updates skipped
Nact  = number of active bounds at final generalized Cauchy point
Projg = norm of the final projected gradient
F     = final function value

           * * *

   N    Tit     Tnf  Tnint  Skip  Nact     Projg        F
*****     18     21      1     0     0   4.598D+14   6.764D+19
  F =   6.7643850898825806E+019

STOP: TOTAL NO. of f AND g EVALUATIONS EXCEEDS LIMIT
Current loss value: 6.764385e+19
Start of iteration 9
RUNNING THE L-BFGS-B CODE

           * * *

Machine precision = 2.220D-16
 N =       841200     M =           10

At X0         0 variables are exactly at the bounds

At iterate    0    f=  6.76439D+19    |proj g|=  4.59754D+14
    \end{Verbatim}

    \begin{Verbatim}[commandchars=\\\{\}]
 This problem is unconstrained.
    \end{Verbatim}

    \begin{Verbatim}[commandchars=\\\{\}]

At iterate    1    f=  6.76102D+19    |proj g|=  3.87343D+14

At iterate    2    f=  6.72512D+19    |proj g|=  4.52570D+14

At iterate    3    f=  6.69816D+19    |proj g|=  6.27490D+14

At iterate    4    f=  6.67442D+19    |proj g|=  1.07344D+15

At iterate    5    f=  6.64766D+19    |proj g|=  6.78219D+14

At iterate    6    f=  6.59665D+19    |proj g|=  5.72283D+14

At iterate    7    f=  6.55110D+19    |proj g|=  5.55712D+14

At iterate    8    f=  6.50789D+19    |proj g|=  1.47385D+15

At iterate    9    f=  6.45808D+19    |proj g|=  4.59516D+14

At iterate   10    f=  6.43927D+19    |proj g|=  4.57898D+14

At iterate   11    f=  6.41410D+19    |proj g|=  4.20263D+14

At iterate   12    f=  6.38077D+19    |proj g|=  4.70625D+14

At iterate   13    f=  6.35074D+19    |proj g|=  2.01695D+15

At iterate   14    f=  6.27223D+19    |proj g|=  6.30264D+14

At iterate   15    f=  6.23469D+19    |proj g|=  5.97304D+14

At iterate   16    f=  6.17754D+19    |proj g|=  8.24868D+14

At iterate   17    f=  6.14865D+19    |proj g|=  2.16567D+15

At iterate   18    f=  6.03941D+19    |proj g|=  9.09125D+14

At iterate   19    f=  5.89155D+19    |proj g|=  1.11271D+15

At iterate   20    f=  5.63995D+19    |proj g|=  1.53639D+15

           * * *

Tit   = total number of iterations
Tnf   = total number of function evaluations
Tnint = total number of segments explored during Cauchy searches
Skip  = number of BFGS updates skipped
Nact  = number of active bounds at final generalized Cauchy point
Projg = norm of the final projected gradient
F     = final function value

           * * *

   N    Tit     Tnf  Tnint  Skip  Nact     Projg        F
*****     20     21      1     0     0   1.536D+15   5.640D+19
  F =   5.6399492927235031E+019

STOP: TOTAL NO. of f AND g EVALUATIONS EXCEEDS LIMIT
Current loss value: 5.6399493e+19
Start of iteration 10
RUNNING THE L-BFGS-B CODE

           * * *

Machine precision = 2.220D-16
 N =       841200     M =           10

At X0         0 variables are exactly at the bounds

At iterate    0    f=  5.63995D+19    |proj g|=  1.53639D+15
    \end{Verbatim}

    \begin{Verbatim}[commandchars=\\\{\}]
 This problem is unconstrained.
    \end{Verbatim}

    \begin{Verbatim}[commandchars=\\\{\}]

At iterate    1    f=  5.58214D+19    |proj g|=  8.63644D+14

At iterate    2    f=  5.52496D+19    |proj g|=  8.15982D+14

At iterate    3    f=  5.49342D+19    |proj g|=  6.74456D+14

At iterate    4    f=  5.43618D+19    |proj g|=  9.25787D+14

At iterate    5    f=  5.41656D+19    |proj g|=  1.84184D+15

At iterate    6    f=  5.38221D+19    |proj g|=  5.70365D+14

At iterate    7    f=  5.36485D+19    |proj g|=  4.66149D+14

At iterate    8    f=  5.34618D+19    |proj g|=  6.95740D+14

At iterate    9    f=  5.31379D+19    |proj g|=  6.40900D+14

At iterate   10    f=  5.26371D+19    |proj g|=  9.99393D+14

At iterate   11    f=  5.23819D+19    |proj g|=  6.80416D+14

At iterate   12    f=  5.21703D+19    |proj g|=  3.86420D+14

At iterate   13    f=  5.19323D+19    |proj g|=  6.79079D+14

At iterate   14    f=  5.17349D+19    |proj g|=  1.05034D+15

At iterate   15    f=  5.15081D+19    |proj g|=  8.93157D+14

At iterate   16    f=  5.12333D+19    |proj g|=  1.13311D+15

At iterate   17    f=  5.10428D+19    |proj g|=  6.95238D+14

At iterate   18    f=  5.09559D+19    |proj g|=  2.82223D+14

           * * *

Tit   = total number of iterations
Tnf   = total number of function evaluations
Tnint = total number of segments explored during Cauchy searches
Skip  = number of BFGS updates skipped
Nact  = number of active bounds at final generalized Cauchy point
Projg = norm of the final projected gradient
F     = final function value

           * * *

   N    Tit     Tnf  Tnint  Skip  Nact     Projg        F
*****     18     21      1     0     0   2.822D+14   5.096D+19
  F =   5.0955854838581166E+019

STOP: TOTAL NO. of f AND g EVALUATIONS EXCEEDS LIMIT
Current loss value: 5.0955855e+19
Start of iteration 11
RUNNING THE L-BFGS-B CODE

           * * *

Machine precision = 2.220D-16
 N =       841200     M =           10

At X0         0 variables are exactly at the bounds

At iterate    0    f=  5.09559D+19    |proj g|=  2.82223D+14
    \end{Verbatim}

    \begin{Verbatim}[commandchars=\\\{\}]
 This problem is unconstrained.
    \end{Verbatim}

    \begin{Verbatim}[commandchars=\\\{\}]

At iterate    1    f=  5.09364D+19    |proj g|=  2.00567D+14

At iterate    2    f=  5.08219D+19    |proj g|=  2.66358D+14

At iterate    3    f=  5.06869D+19    |proj g|=  5.32530D+14

At iterate    4    f=  5.04645D+19    |proj g|=  1.12630D+15

At iterate    5    f=  5.02304D+19    |proj g|=  6.41668D+14

At iterate    6    f=  5.00768D+19    |proj g|=  5.21292D+14

At iterate    7    f=  4.99247D+19    |proj g|=  4.42382D+14

At iterate    8    f=  4.97914D+19    |proj g|=  5.39113D+14

At iterate    9    f=  4.94463D+19    |proj g|=  4.76558D+14

At iterate   10    f=  4.93244D+19    |proj g|=  1.24507D+15

At iterate   11    f=  4.89907D+19    |proj g|=  3.38259D+14

At iterate   12    f=  4.88720D+19    |proj g|=  3.07251D+14

At iterate   13    f=  4.87000D+19    |proj g|=  3.61881D+14

At iterate   14    f=  4.84262D+19    |proj g|=  5.26052D+14

At iterate   15    f=  4.79511D+19    |proj g|=  6.31861D+14

At iterate   16    f=  4.74110D+19    |proj g|=  1.23112D+15

At iterate   17    f=  4.67415D+19    |proj g|=  1.05998D+15

At iterate   18    f=  4.59684D+19    |proj g|=  1.28628D+15

At iterate   19    f=  4.42607D+19    |proj g|=  1.29078D+15

At iterate   20    f=  4.31914D+19    |proj g|=  1.71967D+15

           * * *

Tit   = total number of iterations
Tnf   = total number of function evaluations
Tnint = total number of segments explored during Cauchy searches
Skip  = number of BFGS updates skipped
Nact  = number of active bounds at final generalized Cauchy point
Projg = norm of the final projected gradient
F     = final function value

           * * *

   N    Tit     Tnf  Tnint  Skip  Nact     Projg        F
*****     20     22      1     0     0   1.720D+15   4.319D+19
  F =   4.3191350013831676E+019

STOP: TOTAL NO. of f AND g EVALUATIONS EXCEEDS LIMIT
Current loss value: 4.319135e+19
Start of iteration 12
RUNNING THE L-BFGS-B CODE

           * * *

Machine precision = 2.220D-16
 N =       841200     M =           10

At X0         0 variables are exactly at the bounds

At iterate    0    f=  4.31914D+19    |proj g|=  1.71967D+15
    \end{Verbatim}

    \begin{Verbatim}[commandchars=\\\{\}]
 This problem is unconstrained.
    \end{Verbatim}

    \begin{Verbatim}[commandchars=\\\{\}]

At iterate    1    f=  4.26681D+19    |proj g|=  9.47960D+14

At iterate    2    f=  4.22107D+19    |proj g|=  6.24987D+14

At iterate    3    f=  4.18804D+19    |proj g|=  7.71261D+14

At iterate    4    f=  4.11453D+19    |proj g|=  1.19914D+15

At iterate    5    f=  4.10300D+19    |proj g|=  1.63743D+15

At iterate    6    f=  4.05723D+19    |proj g|=  6.16115D+14

At iterate    7    f=  4.04159D+19    |proj g|=  3.52827D+14

At iterate    8    f=  4.02210D+19    |proj g|=  5.16400D+14

At iterate    9    f=  3.99688D+19    |proj g|=  6.08152D+14

At iterate   10    f=  3.93564D+19    |proj g|=  6.69384D+14

At iterate   11    f=  3.91221D+19    |proj g|=  1.23300D+15

At iterate   12    f=  3.86716D+19    |proj g|=  5.35791D+14

At iterate   13    f=  3.85001D+19    |proj g|=  2.67368D+14

At iterate   14    f=  3.84245D+19    |proj g|=  3.69429D+14

At iterate   15    f=  3.83646D+19    |proj g|=  2.03099D+14

At iterate   16    f=  3.82897D+19    |proj g|=  1.72166D+14

At iterate   17    f=  3.82469D+19    |proj g|=  2.37922D+14

At iterate   18    f=  3.81749D+19    |proj g|=  7.27756D+14

           * * *

Tit   = total number of iterations
Tnf   = total number of function evaluations
Tnint = total number of segments explored during Cauchy searches
Skip  = number of BFGS updates skipped
Nact  = number of active bounds at final generalized Cauchy point
Projg = norm of the final projected gradient
F     = final function value

           * * *

   N    Tit     Tnf  Tnint  Skip  Nact     Projg        F
*****     18     21      1     0     0   7.278D+14   3.817D+19
  F =   3.8174942561312965E+019

STOP: TOTAL NO. of f AND g EVALUATIONS EXCEEDS LIMIT
Current loss value: 3.8174943e+19
Start of iteration 13
RUNNING THE L-BFGS-B CODE

           * * *

Machine precision = 2.220D-16
 N =       841200     M =           10

At X0         0 variables are exactly at the bounds

At iterate    0    f=  3.81749D+19    |proj g|=  7.27756D+14
    \end{Verbatim}

    \begin{Verbatim}[commandchars=\\\{\}]
 This problem is unconstrained.
    \end{Verbatim}

    \begin{Verbatim}[commandchars=\\\{\}]

At iterate    1    f=  3.81268D+19    |proj g|=  5.10873D+14

At iterate    2    f=  3.80650D+19    |proj g|=  2.20817D+14

At iterate    3    f=  3.80464D+19    |proj g|=  2.71903D+14

At iterate    4    f=  3.79780D+19    |proj g|=  3.04414D+14

At iterate    5    f=  3.79222D+19    |proj g|=  2.19560D+14

At iterate    6    f=  3.78541D+19    |proj g|=  1.78189D+14

At iterate    7    f=  3.77975D+19    |proj g|=  4.64000D+14

At iterate    8    f=  3.77193D+19    |proj g|=  4.63526D+14

At iterate    9    f=  3.75647D+19    |proj g|=  3.75639D+14

At iterate   10    f=  3.74103D+19    |proj g|=  6.32466D+14

At iterate   11    f=  3.71992D+19    |proj g|=  3.63133D+14

At iterate   12    f=  3.70020D+19    |proj g|=  3.02001D+14

At iterate   13    f=  3.68338D+19    |proj g|=  3.69774D+14

At iterate   14    f=  3.65408D+19    |proj g|=  7.66882D+14

At iterate   15    f=  3.62720D+19    |proj g|=  4.03377D+14

At iterate   16    f=  3.61088D+19    |proj g|=  3.68810D+14

At iterate   17    f=  3.58934D+19    |proj g|=  5.12827D+14

At iterate   18    f=  3.56817D+19    |proj g|=  4.62277D+14

At iterate   19    f=  3.51103D+19    |proj g|=  7.42542D+14

At iterate   20    f=  3.45135D+19    |proj g|=  1.48844D+15

           * * *

Tit   = total number of iterations
Tnf   = total number of function evaluations
Tnint = total number of segments explored during Cauchy searches
Skip  = number of BFGS updates skipped
Nact  = number of active bounds at final generalized Cauchy point
Projg = norm of the final projected gradient
F     = final function value

           * * *

   N    Tit     Tnf  Tnint  Skip  Nact     Projg        F
*****     20     21      1     0     0   1.488D+15   3.451D+19
  F =   3.4513491875004940E+019

STOP: TOTAL NO. of f AND g EVALUATIONS EXCEEDS LIMIT
Current loss value: 3.4513492e+19
Start of iteration 14
    \end{Verbatim}

    \begin{Verbatim}[commandchars=\\\{\}]
 This problem is unconstrained.
    \end{Verbatim}

    \begin{Verbatim}[commandchars=\\\{\}]
RUNNING THE L-BFGS-B CODE

           * * *

Machine precision = 2.220D-16
 N =       841200     M =           10

At X0         0 variables are exactly at the bounds

At iterate    0    f=  3.45135D+19    |proj g|=  1.48844D+15

At iterate    1    f=  3.44146D+19    |proj g|=  1.25761D+15

At iterate    2    f=  3.40357D+19    |proj g|=  4.85347D+14

At iterate    3    f=  3.39104D+19    |proj g|=  5.33456D+14

At iterate    4    f=  3.34496D+19    |proj g|=  8.42379D+14

At iterate    5    f=  3.33790D+19    |proj g|=  7.68679D+14

At iterate    6    f=  3.32554D+19    |proj g|=  3.63680D+14

At iterate    7    f=  3.31745D+19    |proj g|=  2.25801D+14

At iterate    8    f=  3.30910D+19    |proj g|=  3.62515D+14

At iterate    9    f=  3.29830D+19    |proj g|=  4.19462D+14

At iterate   10    f=  3.28375D+19    |proj g|=  4.32463D+14

At iterate   11    f=  3.27682D+19    |proj g|=  4.52773D+14

At iterate   12    f=  3.26758D+19    |proj g|=  1.96724D+14

At iterate   13    f=  3.26282D+19    |proj g|=  2.30669D+14

At iterate   14    f=  3.25350D+19    |proj g|=  3.02578D+14

At iterate   15    f=  3.23819D+19    |proj g|=  3.37760D+14

At iterate   16    f=  3.23410D+19    |proj g|=  5.38122D+14

At iterate   17    f=  3.22388D+19    |proj g|=  1.91344D+14

At iterate   18    f=  3.21928D+19    |proj g|=  1.80946D+14

           * * *

Tit   = total number of iterations
Tnf   = total number of function evaluations
Tnint = total number of segments explored during Cauchy searches
Skip  = number of BFGS updates skipped
Nact  = number of active bounds at final generalized Cauchy point
Projg = norm of the final projected gradient
F     = final function value

           * * *

   N    Tit     Tnf  Tnint  Skip  Nact     Projg        F
*****     18     21      1     0     0   1.809D+14   3.219D+19
  F =   3.2192779070537204E+019

STOP: TOTAL NO. of f AND g EVALUATIONS EXCEEDS LIMIT
Current loss value: 3.219278e+19
Start of iteration 15
RUNNING THE L-BFGS-B CODE

           * * *

Machine precision = 2.220D-16
 N =       841200     M =           10

At X0         0 variables are exactly at the bounds

At iterate    0    f=  3.21928D+19    |proj g|=  1.80946D+14
    \end{Verbatim}

    \begin{Verbatim}[commandchars=\\\{\}]
 This problem is unconstrained.
    \end{Verbatim}

    \begin{Verbatim}[commandchars=\\\{\}]

At iterate    1    f=  3.21811D+19    |proj g|=  1.40607D+14

At iterate    2    f=  3.21625D+19    |proj g|=  2.01745D+14

At iterate    3    f=  3.21208D+19    |proj g|=  2.72846D+14

At iterate    4    f=  3.20638D+19    |proj g|=  3.24981D+14

At iterate    5    f=  3.20102D+19    |proj g|=  6.17449D+14

At iterate    6    f=  3.19372D+19    |proj g|=  2.02273D+14

At iterate    7    f=  3.18833D+19    |proj g|=  2.65469D+14

At iterate    8    f=  3.18219D+19    |proj g|=  3.11273D+14

At iterate    9    f=  3.16739D+19    |proj g|=  3.11339D+14

At iterate   10    f=  3.16194D+19    |proj g|=  5.16174D+14

At iterate   11    f=  3.15270D+19    |proj g|=  2.01324D+14

At iterate   12    f=  3.14680D+19    |proj g|=  1.87196D+14

At iterate   13    f=  3.13951D+19    |proj g|=  4.85196D+14

At iterate   14    f=  3.13174D+19    |proj g|=  2.97828D+14

At iterate   15    f=  3.12214D+19    |proj g|=  2.64409D+14

At iterate   16    f=  3.11019D+19    |proj g|=  3.32715D+14

At iterate   17    f=  3.09167D+19    |proj g|=  4.54613D+14

At iterate   18    f=  3.06672D+19    |proj g|=  4.28152D+14

At iterate   19    f=  3.02154D+19    |proj g|=  7.16448D+14

           * * *

Tit   = total number of iterations
Tnf   = total number of function evaluations
Tnint = total number of segments explored during Cauchy searches
Skip  = number of BFGS updates skipped
Nact  = number of active bounds at final generalized Cauchy point
Projg = norm of the final projected gradient
F     = final function value

           * * *

   N    Tit     Tnf  Tnint  Skip  Nact     Projg        F
*****     19     21      1     0     0   7.164D+14   3.022D+19
  F =   3.0215430553284379E+019

STOP: TOTAL NO. of f AND g EVALUATIONS EXCEEDS LIMIT
Current loss value: 3.021543e+19
Start of iteration 16
RUNNING THE L-BFGS-B CODE

           * * *

Machine precision = 2.220D-16
 N =       841200     M =           10

At X0         0 variables are exactly at the bounds

At iterate    0    f=  3.02154D+19    |proj g|=  7.16448D+14
    \end{Verbatim}

    \begin{Verbatim}[commandchars=\\\{\}]
 This problem is unconstrained.
    \end{Verbatim}

    \begin{Verbatim}[commandchars=\\\{\}]

At iterate    1    f=  3.01629D+19    |proj g|=  5.46552D+14

At iterate    2    f=  2.99862D+19    |proj g|=  4.63688D+14

At iterate    3    f=  2.98701D+19    |proj g|=  6.38315D+14

At iterate    4    f=  2.96316D+19    |proj g|=  6.42501D+14

At iterate    5    f=  2.95584D+19    |proj g|=  6.88080D+14

At iterate    6    f=  2.94626D+19    |proj g|=  3.56632D+14

At iterate    7    f=  2.93676D+19    |proj g|=  3.82295D+14

At iterate    8    f=  2.92863D+19    |proj g|=  3.84612D+14

At iterate    9    f=  2.91763D+19    |proj g|=  2.38163D+14

At iterate   10    f=  2.91391D+19    |proj g|=  6.86237D+14

At iterate   11    f=  2.90259D+19    |proj g|=  2.73089D+14

At iterate   12    f=  2.89870D+19    |proj g|=  2.94820D+14

At iterate   13    f=  2.89123D+19    |proj g|=  2.44142D+14

At iterate   14    f=  2.88186D+19    |proj g|=  3.34198D+14

At iterate   15    f=  2.87711D+19    |proj g|=  3.42080D+14

At iterate   16    f=  2.87221D+19    |proj g|=  1.35634D+14

At iterate   17    f=  2.86930D+19    |proj g|=  1.27512D+14

At iterate   18    f=  2.86660D+19    |proj g|=  4.43239D+14

           * * *

Tit   = total number of iterations
Tnf   = total number of function evaluations
Tnint = total number of segments explored during Cauchy searches
Skip  = number of BFGS updates skipped
Nact  = number of active bounds at final generalized Cauchy point
Projg = norm of the final projected gradient
F     = final function value

           * * *

   N    Tit     Tnf  Tnint  Skip  Nact     Projg        F
*****     18     21      1     0     0   4.432D+14   2.867D+19
  F =   2.8665950388910817E+019

STOP: TOTAL NO. of f AND g EVALUATIONS EXCEEDS LIMIT
Current loss value: 2.866595e+19
Start of iteration 17
RUNNING THE L-BFGS-B CODE

           * * *

Machine precision = 2.220D-16
 N =       841200     M =           10

At X0         0 variables are exactly at the bounds

At iterate    0    f=  2.86660D+19    |proj g|=  4.43239D+14
    \end{Verbatim}

    \begin{Verbatim}[commandchars=\\\{\}]
 This problem is unconstrained.
    \end{Verbatim}

    \begin{Verbatim}[commandchars=\\\{\}]

At iterate    1    f=  2.86449D+19    |proj g|=  1.63787D+14

At iterate    2    f=  2.86345D+19    |proj g|=  1.20229D+14

At iterate    3    f=  2.86231D+19    |proj g|=  1.64450D+14

At iterate    4    f=  2.85974D+19    |proj g|=  2.56646D+14

At iterate    5    f=  2.85623D+19    |proj g|=  2.28604D+14

At iterate    6    f=  2.85530D+19    |proj g|=  6.78870D+14

At iterate    7    f=  2.84796D+19    |proj g|=  1.79186D+14

At iterate    8    f=  2.84504D+19    |proj g|=  1.82660D+14

At iterate    9    f=  2.84062D+19    |proj g|=  2.69346D+14

At iterate   10    f=  2.83444D+19    |proj g|=  2.96242D+14

At iterate   11    f=  2.82805D+19    |proj g|=  5.33666D+14

At iterate   12    f=  2.82043D+19    |proj g|=  2.00730D+14

At iterate   13    f=  2.81667D+19    |proj g|=  2.46865D+14

At iterate   14    f=  2.81011D+19    |proj g|=  2.54583D+14

At iterate   15    f=  2.80086D+19    |proj g|=  5.69209D+14

At iterate   16    f=  2.78731D+19    |proj g|=  3.59818D+14

At iterate   17    f=  2.77723D+19    |proj g|=  2.81471D+14

At iterate   18    f=  2.76725D+19    |proj g|=  4.01382D+14

At iterate   19    f=  2.75270D+19    |proj g|=  3.82244D+14

At iterate   20    f=  2.72935D+19    |proj g|=  5.11684D+14

           * * *

Tit   = total number of iterations
Tnf   = total number of function evaluations
Tnint = total number of segments explored during Cauchy searches
Skip  = number of BFGS updates skipped
Nact  = number of active bounds at final generalized Cauchy point
Projg = norm of the final projected gradient
F     = final function value

           * * *

   N    Tit     Tnf  Tnint  Skip  Nact     Projg        F
*****     20     21      1     0     0   5.117D+14   2.729D+19
  F =   2.7293467407353381E+019

STOP: TOTAL NO. of f AND g EVALUATIONS EXCEEDS LIMIT
Current loss value: 2.7293467e+19
Start of iteration 18
RUNNING THE L-BFGS-B CODE

           * * *

Machine precision = 2.220D-16
 N =       841200     M =           10

At X0         0 variables are exactly at the bounds

At iterate    0    f=  2.72935D+19    |proj g|=  5.11684D+14
    \end{Verbatim}

    \begin{Verbatim}[commandchars=\\\{\}]
 This problem is unconstrained.
    \end{Verbatim}

    \begin{Verbatim}[commandchars=\\\{\}]

At iterate    1    f=  2.72585D+19    |proj g|=  3.65103D+14

At iterate    2    f=  2.71652D+19    |proj g|=  3.69120D+14

At iterate    3    f=  2.70774D+19    |proj g|=  5.14481D+14

At iterate    4    f=  2.69311D+19    |proj g|=  3.95743D+14

At iterate    5    f=  2.68337D+19    |proj g|=  6.31266D+14

At iterate    6    f=  2.67194D+19    |proj g|=  3.23455D+14

At iterate    7    f=  2.66485D+19    |proj g|=  2.91777D+14

At iterate    8    f=  2.65976D+19    |proj g|=  3.84347D+14

At iterate    9    f=  2.65395D+19    |proj g|=  2.34804D+14

At iterate   10    f=  2.64747D+19    |proj g|=  3.20143D+14

At iterate   11    f=  2.63722D+19    |proj g|=  3.85313D+14

At iterate   12    f=  2.63370D+19    |proj g|=  3.13545D+14

At iterate   13    f=  2.62957D+19    |proj g|=  1.77133D+14

At iterate   14    f=  2.62454D+19    |proj g|=  1.46565D+14

At iterate   15    f=  2.62193D+19    |proj g|=  2.63700D+14

At iterate   16    f=  2.62024D+19    |proj g|=  1.71383D+14

At iterate   17    f=  2.61893D+19    |proj g|=  9.91390D+13

At iterate   18    f=  2.61748D+19    |proj g|=  1.63970D+14

At iterate   19    f=  2.61513D+19    |proj g|=  2.08777D+14

           * * *

Tit   = total number of iterations
Tnf   = total number of function evaluations
Tnint = total number of segments explored during Cauchy searches
Skip  = number of BFGS updates skipped
Nact  = number of active bounds at final generalized Cauchy point
Projg = norm of the final projected gradient
F     = final function value

           * * *

   N    Tit     Tnf  Tnint  Skip  Nact     Projg        F
*****     19     21      1     0     0   2.088D+14   2.615D+19
  F =   2.6151250747954561E+019

STOP: TOTAL NO. of f AND g EVALUATIONS EXCEEDS LIMIT
Current loss value: 2.615125e+19
Start of iteration 19
RUNNING THE L-BFGS-B CODE

           * * *

Machine precision = 2.220D-16
 N =       841200     M =           10

At X0         0 variables are exactly at the bounds

At iterate    0    f=  2.61513D+19    |proj g|=  2.08777D+14
    \end{Verbatim}

    \begin{Verbatim}[commandchars=\\\{\}]
 This problem is unconstrained.
    \end{Verbatim}

    \begin{Verbatim}[commandchars=\\\{\}]

At iterate    1    f=  2.61420D+19    |proj g|=  1.15913D+14

At iterate    2    f=  2.61361D+19    |proj g|=  1.49954D+14

At iterate    3    f=  2.61139D+19    |proj g|=  1.79306D+14

At iterate    4    f=  2.60981D+19    |proj g|=  2.60210D+14

At iterate    5    f=  2.60740D+19    |proj g|=  2.02262D+14

At iterate    6    f=  2.60139D+19    |proj g|=  2.77197D+14

At iterate    7    f=  2.60009D+19    |proj g|=  5.49328D+14

At iterate    8    f=  2.59624D+19    |proj g|=  2.04768D+14

At iterate    9    f=  2.59415D+19    |proj g|=  1.46580D+14

At iterate   10    f=  2.59082D+19    |proj g|=  2.46696D+14

At iterate   11    f=  2.58498D+19    |proj g|=  3.74120D+14

At iterate   12    f=  2.57345D+19    |proj g|=  4.20505D+14

At iterate   13    f=  2.56556D+19    |proj g|=  8.53551D+14

At iterate   14    f=  2.55291D+19    |proj g|=  4.35192D+14

At iterate   15    f=  2.54101D+19    |proj g|=  2.88393D+14

At iterate   16    f=  2.53314D+19    |proj g|=  3.00418D+14

At iterate   17    f=  2.51849D+19    |proj g|=  3.33506D+14

At iterate   18    f=  2.50220D+19    |proj g|=  7.08309D+14

At iterate   19    f=  2.47921D+19    |proj g|=  4.18376D+14

           * * *

Tit   = total number of iterations
Tnf   = total number of function evaluations
Tnint = total number of segments explored during Cauchy searches
Skip  = number of BFGS updates skipped
Nact  = number of active bounds at final generalized Cauchy point
Projg = norm of the final projected gradient
F     = final function value

           * * *

   N    Tit     Tnf  Tnint  Skip  Nact     Projg        F
*****     19     21      1     0     0   4.184D+14   2.479D+19
  F =   2.4792074056116470E+019

STOP: TOTAL NO. of f AND g EVALUATIONS EXCEEDS LIMIT
Current loss value: 2.4792074e+19
    \end{Verbatim}

    \begin{tcolorbox}[breakable, size=fbox, boxrule=1pt, pad at break*=1mm,colback=cellbackground, colframe=cellborder]
\prompt{In}{incolor}{229}{\boxspacing}
\begin{Verbatim}[commandchars=\\\{\}]
\PY{c+c1}{\PYZsh{} save current generated image}
\PY{n}{imgx} \PY{o}{=} \PY{n}{deprocess\PYZus{}image}\PY{p}{(}\PY{n}{best\PYZus{}img}\PY{o}{.}\PY{n}{copy}\PY{p}{(}\PY{p}{)}\PY{p}{)}
\PY{n}{plt}\PY{o}{.}\PY{n}{imshow}\PY{p}{(}\PY{n}{imgx}\PY{p}{)}
\end{Verbatim}
\end{tcolorbox}

            \begin{tcolorbox}[breakable, size=fbox, boxrule=.5pt, pad at break*=1mm, opacityfill=0]
\prompt{Out}{outcolor}{229}{\boxspacing}
\begin{Verbatim}[commandchars=\\\{\}]
<matplotlib.image.AxesImage at 0x2c5f137f0>
\end{Verbatim}
\end{tcolorbox}
        
    \begin{center}
    \adjustimage{max size={0.9\linewidth}{0.9\paperheight}}{output_25_1.png}
    \end{center}
    { \hspace*{\fill} \\}
    
    \begin{tcolorbox}[breakable, size=fbox, boxrule=1pt, pad at break*=1mm,colback=cellbackground, colframe=cellborder]
\prompt{In}{incolor}{230}{\boxspacing}
\begin{Verbatim}[commandchars=\\\{\}]
\PY{n}{plt}\PY{o}{.}\PY{n}{figure}\PY{p}{(}\PY{n}{figsize}\PY{o}{=}\PY{p}{(}\PY{l+m+mi}{30}\PY{p}{,}\PY{l+m+mi}{30}\PY{p}{)}\PY{p}{)}
\PY{n}{plt}\PY{o}{.}\PY{n}{subplot}\PY{p}{(}\PY{l+m+mi}{5}\PY{p}{,}\PY{l+m+mi}{5}\PY{p}{,}\PY{l+m+mi}{1}\PY{p}{)}
\PY{n}{plt}\PY{o}{.}\PY{n}{title}\PY{p}{(}\PY{l+s+s2}{\PYZdq{}}\PY{l+s+s2}{Base Image}\PY{l+s+s2}{\PYZdq{}}\PY{p}{,}\PY{n}{fontsize}\PY{o}{=}\PY{l+m+mi}{20}\PY{p}{)}
\PY{n}{img\PYZus{}base} \PY{o}{=} \PY{n}{image\PYZus{}utils}\PY{o}{.}\PY{n}{load\PYZus{}img}\PY{p}{(}\PY{n}{base\PYZus{}image\PYZus{}path}\PY{p}{)}
\PY{n}{plt}\PY{o}{.}\PY{n}{imshow}\PY{p}{(}\PY{n}{img\PYZus{}base}\PY{p}{)}

\PY{n}{plt}\PY{o}{.}\PY{n}{subplot}\PY{p}{(}\PY{l+m+mi}{5}\PY{p}{,}\PY{l+m+mi}{5}\PY{p}{,}\PY{l+m+mi}{1}\PY{o}{+}\PY{l+m+mi}{1}\PY{p}{)}
\PY{n}{plt}\PY{o}{.}\PY{n}{title}\PY{p}{(}\PY{l+s+s2}{\PYZdq{}}\PY{l+s+s2}{Style Image}\PY{l+s+s2}{\PYZdq{}}\PY{p}{,}\PY{n}{fontsize}\PY{o}{=}\PY{l+m+mi}{20}\PY{p}{)}
\PY{n}{img\PYZus{}style} \PY{o}{=} \PY{n}{image\PYZus{}utils}\PY{o}{.}\PY{n}{load\PYZus{}img}\PY{p}{(}\PY{n}{style\PYZus{}image\PYZus{}path}\PY{p}{)}
\PY{n}{plt}\PY{o}{.}\PY{n}{imshow}\PY{p}{(}\PY{n}{img\PYZus{}style}\PY{p}{)}

\PY{n}{plt}\PY{o}{.}\PY{n}{subplot}\PY{p}{(}\PY{l+m+mi}{5}\PY{p}{,}\PY{l+m+mi}{5}\PY{p}{,}\PY{l+m+mi}{1}\PY{o}{+}\PY{l+m+mi}{2}\PY{p}{)}
\PY{n}{plt}\PY{o}{.}\PY{n}{title}\PY{p}{(}\PY{l+s+s2}{\PYZdq{}}\PY{l+s+s2}{Final Image}\PY{l+s+s2}{\PYZdq{}}\PY{p}{,}\PY{n}{fontsize}\PY{o}{=}\PY{l+m+mi}{20}\PY{p}{)}
\PY{n}{plt}\PY{o}{.}\PY{n}{imshow}\PY{p}{(}\PY{n}{imgx}\PY{p}{)}
\end{Verbatim}
\end{tcolorbox}

            \begin{tcolorbox}[breakable, size=fbox, boxrule=.5pt, pad at break*=1mm, opacityfill=0]
\prompt{Out}{outcolor}{230}{\boxspacing}
\begin{Verbatim}[commandchars=\\\{\}]
<matplotlib.image.AxesImage at 0x2bff48220>
\end{Verbatim}
\end{tcolorbox}
        
    \begin{center}
    \adjustimage{max size={0.9\linewidth}{0.9\paperheight}}{output_26_1.png}
    \end{center}
    { \hspace*{\fill} \\}
    
    \begin{tcolorbox}[breakable, size=fbox, boxrule=1pt, pad at break*=1mm,colback=cellbackground, colframe=cellborder]
\prompt{In}{incolor}{ }{\boxspacing}
\begin{Verbatim}[commandchars=\\\{\}]

\end{Verbatim}
\end{tcolorbox}

    \begin{tcolorbox}[breakable, size=fbox, boxrule=1pt, pad at break*=1mm,colback=cellbackground, colframe=cellborder]
\prompt{In}{incolor}{ }{\boxspacing}
\begin{Verbatim}[commandchars=\\\{\}]

\end{Verbatim}
\end{tcolorbox}

    \begin{tcolorbox}[breakable, size=fbox, boxrule=1pt, pad at break*=1mm,colback=cellbackground, colframe=cellborder]
\prompt{In}{incolor}{ }{\boxspacing}
\begin{Verbatim}[commandchars=\\\{\}]

\end{Verbatim}
\end{tcolorbox}

    \begin{tcolorbox}[breakable, size=fbox, boxrule=1pt, pad at break*=1mm,colback=cellbackground, colframe=cellborder]
\prompt{In}{incolor}{ }{\boxspacing}
\begin{Verbatim}[commandchars=\\\{\}]

\end{Verbatim}
\end{tcolorbox}

    \begin{tcolorbox}[breakable, size=fbox, boxrule=1pt, pad at break*=1mm,colback=cellbackground, colframe=cellborder]
\prompt{In}{incolor}{ }{\boxspacing}
\begin{Verbatim}[commandchars=\\\{\}]

\end{Verbatim}
\end{tcolorbox}


    % Add a bibliography block to the postdoc
    
    
    
\end{document}
